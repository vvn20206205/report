Trước khi kiến trúc vi dịch vụ trở nên phổ biến, kiến trúc nguyên khối đã được áp dụng rộng rãi trong kiến trúc phần mềm truyền thống. Kiến trúc nguyên khối là kiến trúc phần mềm trong đó tất cả các thành phần của dự án được xây dựng thành một đơn vị triển khai duy nhất.

Trong kiến trúc nguyên khối, bất kỳ thay đổi nào đối với một thành phần đều yêu cầu toàn bộ dự án phải được kiểm thử và triển khai lại dẫn đến tốc độ phát triển chậm và thiếu khả năng mở rộng.

Ví dụ: Mô hình MVC (Model - View - Controller) là một trong những dạng của kiến trúc nguyên khối. Trong mô hình này, ứng dụng được chia thành ba thành phần chính:

\begin{itemize}

\item \textbf{Mô hình (Model):} Đại diện cho dữ liệu và logic xử lý dữ liệu.

\item \textbf{Giao diện (View):} Đại diện cho giao diện người dùng.

\item \textbf{Bộ điều khiển (Controller):} Nhận yêu cầu người dùng thông qua View, sau đó tương tác với Model để làm việc với dữ liệu.

\end{itemize}
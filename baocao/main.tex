% \documentclass{article}
% \documentclass{book}
\documentclass{report} % Chọn cỡ chữ
\usepackage{Start}
\begin{document} % Bắt đầu

% \begin{titlepage}

    % Vẽ hình chữ nhật
    
    \begin{tikzpicture}[remember picture, overlay]\draw [line width = 3pt]($ (current page.north west) + (3.0cm, - 2.5cm)$)rectangle($ (current page.south east) + (- 2.5cm, 2.5cm)$);\draw [line width = 0.5pt]($ (current page.north west) + (3.1cm, - 2.6cm)$)rectangle($ (current page.south east) + (- 2.6cm, 2.6cm)$);\end{tikzpicture}
    
    \begin{center}
    
    \vspace{- 0.4cm}
    
    \textbf{ĐẠI HỌC BÁCH KHOA HÀ NỘI} \\
    
    \textbf{VIỆN TOÁN ỨNG DỤNG VÀ TIN HỌC} \\
    
    \textbf{******}
    
    \vspace{0.8cm}
    
    \begin{figure}[H]
    
    \centering
    
    \includegraphics[scale = 0.5]{pictures/_hust/main.png}
    
    \end{figure}
    
    \vspace{0.7cm}
    
    \textbf{\fontsize{16pt}{30pt}\selectfont {BÁO CÁO ĐỒ ÁN II}}
    
    \vspace{1cm}
    
    \textbf{\fontsize{16pt}{30pt}\selectfont {ĐỀ TÀI:}} \\
    
    \textbf{\fontsize{20pt}{24pt}\selectfont {Sử dụng thiết kế hướng miền \\ xây dựng kiến trúc vi dịch vụ cho \\ bài toán hóa đơn điện tử}} \\
    
    \end{center}
    
    \vspace{0.3cm}
    
    \begin{center}
    
    \textbf{\fontsize{10pt}{24pt}\selectfont {Chuyên ngành: Toán Tin}}
    
    \end{center}
    
    \vspace{0.7cm}
    
    \hspace{3cm}\begin{minipage}{0.7\textwidth}
    
    \begin{tabular}{l l l}
    
    \textbf{\fontsize{10pt}{24pt}\selectfont {Giảng viên hướng dẫn}} & \textbf{\fontsize{10pt}{24pt}\selectfont {TS. Vũ Thành Nam}} \\
    
    \textbf{\fontsize{10pt}{24pt}\selectfont {Sinh viên thực hiện}} & \textbf{\fontsize{10pt}{24pt}\selectfont {Vũ Văn Nghĩa}} \\
    
    \textbf{\fontsize{10pt}{24pt}\selectfont {Mã số sinh viên}} & \textbf{\fontsize{10pt}{24pt}\selectfont {20206205}} \\
    
    \textbf{\fontsize{10pt}{24pt}\selectfont {Lớp}} & \textbf{\fontsize{10pt}{24pt}\selectfont {Toán Tin 02 - K65}} \\
    
    \end{tabular}
    
    \end{minipage}
    
    \vfill
    
    \begin{center}
    
    \textbf{Hà Nội, \the\year}
    
    % \textbf{Hà Nội, \the\month~/~\the\year}
    
    % \textbf{Hà Nội, \the\month~-~\the\year}
    
    \end{center}
    
    \end{titlepage}
    
    
% \begin{titlepage}

    % Vẽ hình chữ nhật
    
    \begin{tikzpicture}[remember picture, overlay]\draw [line width = 3pt]($ (current page.north west) + (3.0cm, - 2.5cm)$)rectangle($ (current page.south east) + (- 2.5cm, 2.5cm)$);\draw [line width = 0.5pt]($ (current page.north west) + (3.1cm, - 2.6cm)$)rectangle($ (current page.south east) + (- 2.6cm, 2.6cm)$);\end{tikzpicture}
    
    \begin{center}
    
    \vspace{- 0.4cm}
    
    \textbf{ĐẠI HỌC BÁCH KHOA HÀ NỘI} \\
    
    \textbf{VIỆN TOÁN ỨNG DỤNG VÀ TIN HỌC} \\
    
    \textbf{******}
    
    \vspace{0.8cm}
    
    \begin{figure}[H]
    
    \centering
    
    \includegraphics[scale = 0.5]{pictures/_hust/main.png}
    
    \end{figure}
    
    \vspace{0.7cm}
    
    \textbf{\fontsize{16pt}{30pt}\selectfont {BÁO CÁO ĐỒ ÁN II}}
    
    \vspace{1cm}
    
    \textbf{\fontsize{16pt}{30pt}\selectfont {ĐỀ TÀI:}} \\
    
    \textbf{\fontsize{20pt}{24pt}\selectfont {Sử dụng thiết kế hướng miền \\ xây dựng kiến trúc vi dịch vụ cho \\ bài toán hóa đơn điện tử}} \\
    
    \end{center}
    
    \vspace{0.3cm}
    
    \begin{center}
    
    \textbf{\fontsize{10pt}{24pt}\selectfont {Chuyên ngành: Toán Tin}}
    
    \end{center}
    
    \vspace{0.7cm}
    
    \hspace{3cm}\begin{minipage}{0.7\textwidth}
    
    \begin{tabular}{l l l}
    
    \textbf{\fontsize{10pt}{24pt}\selectfont {Giảng viên hướng dẫn}} & \textbf{\fontsize{10pt}{24pt}\selectfont {TS. Vũ Thành Nam}} \\
    
    \textbf{\fontsize{10pt}{24pt}\selectfont {Sinh viên thực hiện}} & \textbf{\fontsize{10pt}{24pt}\selectfont {Vũ Văn Nghĩa}} \\
    
    \textbf{\fontsize{10pt}{24pt}\selectfont {Mã số sinh viên}} & \textbf{\fontsize{10pt}{24pt}\selectfont {20206205}} \\
    
    \textbf{\fontsize{10pt}{24pt}\selectfont {Lớp}} & \textbf{\fontsize{10pt}{24pt}\selectfont {Toán Tin 02 - K65}} \\
    
    \end{tabular}
    
    \end{minipage}
    
    \vfill
    
    \begin{center}
    
    \textbf{Hà Nội, \the\year}
    
    % \textbf{Hà Nội, \the\month~/~\the\year}
    
    % \textbf{Hà Nội, \the\month~-~\the\year}
    
    \end{center}
    
    \end{titlepage}
    
    
% \begin{center}

{\bfseries NHẬN XÉT CỦA GIẢNG VIÊN HƯỚNG DẪN}

\end{center}

\begin{enumerate}

\item Mục đích và nội dung của đồ án:

\vspace{20ex}

\item Kết quả đạt được:

\vspace{20ex}

\item Ý thức làm việc của sinh viên:

\vspace{20ex}

\end{enumerate}

\hspace{0.4\textwidth}\begin{minipage}{0.5\textwidth}

\noindent\begin{center}

\textit{Hà Nội, \today} \\

\textbf{Giảng viên hướng dẫn} \\

\textit{(Ký và ghi rõ họ tên)}

\vspace{2cm}

\textbf{TS. Vũ Thành Nam}

\end{center}

\end{minipage}

\pagestyle{empty}
% \includepdf[pages = -]{contents/bao_cao_tien_do/lan_1.pdf}
% \includepdf[pages = -]{contents/bao_cao_tien_do/lan_2.pdf}
%! API gateway
%! Service mesh(1, 2, 3, \dots)
%! CQRS
%! event
%! Saga

% \input{contents/p0_muc_luc}
% \chapter*{\centering LỜI CẢM ƠN}

\addcontentsline{toc}{chapter}{LỜI CẢM ƠN}

Trước hết, em xin gửi lời cảm ơn chân thành và sâu sắc đến TS. Vũ Thành Nam, người thầy đã tận tình hỗ trợ và hướng dẫn em suốt thời gian thực hiện đồ án. Những kiến thức và kinh nghiệm mà em đã tiếp thu được trong quá trình này sẽ đóng góp quan trọng vào sự phát triển và thành công của em trong tương lai. Em xin gửi lời chúc sức khỏe tốt nhất đến thầy, hy vọng thầy luôn dồi dào sức khỏe, đam mê và nhiệt huyết trong công việc giảng dạy.

Em cũng xin gửi lời cảm ơn tới các thầy cô giảng viên trong \emph{"Viện Toán ứng dụng và Tin học"} đã tận tình truyền đạt những kiến thức quý báu cho em. Những kiến thức này không chỉ giúp em phát triển về mặt tri thức mà còn nuôi dưỡng kỹ năng và đam mê trong quá trình học tập và nghiên cứu.

Trong quá trình hoàn thành bài báo cáo đồ án này không tránh khỏi những thiếu sót. Vì vậy, em mong nhận được sự giúp đỡ và ý kiến đóng góp chân thành từ các thầy cô để em có thể cải thiện một cách tốt nhất.

\emph{Em xin chân thành cảm ơn!}

\vspace{0.7cm}

\hspace{0.4\textwidth}\begin{minipage}{0.5\textwidth}

\noindent\begin{center}

\textit{Hà Nội, \today} \\

\vspace{0.5cm}

\textbf{Tác giả} \\

\vspace{0.5cm}

\textbf{Vũ Văn Nghĩa}

\end{center}

\end{minipage}
% \chapter*{\centering DANH SÁCH BẢNG}

\addcontentsline{toc}{chapter}{DANH SÁCH BẢNG}

\makeatletter

\renewcommand\listoftables{

\@starttoc{lot}

}

\makeatother

\listoftables
% \input{contents/p0_danh_sach_hinh_anh}
% %%%%%%%%%%%%%%%%%%%%%%%%%%%%%%

%%%%%%%%%%%%%%%%%%%%%%%%%%%%%%

\chapter*{\centering DANH SÁCH CÁC CỤM TỪ VIẾT TẮT}

\addcontentsline{toc}{chapter}{DANH SÁCH CÁC CỤM TỪ VIẾT TẮT}

% @sau

% @sau

% @sau

% @sau

% @sau

% @sau

% @sau

% @sau

% @sau

% @sau

% @sau

% @sau

% @sau

% @sau

% @sau

% @sau

% @sau

% @sau

% @sau

% @sau

% @sau

% @sau

% @sau

% @sau

% @sau

% @sau

% @sau

% @sau

% @sau

% @sau

\begin{table}[h]

\centering

\begin{tabular}{|c|c|c|c|}

\hline

STT & Từ viết tắt & Từ viết đầy đủ & Mô tả \\

\hline

Dong1 & Dong1 & Cot1 & Cot2 \\

\hline

Dong2 & Dong2 & Cot1 & Cot2 \\

\hline

\end{tabular}

\end{table}

% API; Application Programming Interface; Giao diện lập trình ứng dụng

% CI/CD; Continuous Integration (CI) and Continuous Delivery (CD) ; Quá trình tích hợp và chuyển giao liên tục

% thiết kế hướng miền ; thiết kế hướng miền; Kỹ thuật thiết kế theo hướng miền

% DI; Dependency Injection; Cơ chế tiêm sự phụ thuộc giữa các đối tượng

% HTTP; Hypertext Transfer Protocol; Giao thức truyền tải siêu văn bản

% JSON; JavaScript Object Notation; Một kiểu dữ liệu mở rộng của JavaScript

% ORM; Object Relational Mapping; Một kỹ thuật ánh xạ các đối tượng lập trình với từng bảng trong CSDL quan hệ

% Cơ sở dữ liệu ; CSDL ;

% Tạo (Create), Đọc (Read), Sửa (Update), Xóa (Delete) ; CRUD ;

% Kubernetes ; K8s ; kubernetes

% Số điện thoại ; SĐT ;

% UML

% MVC; Model View Controller; Một mẫu thiết kế ứng dụng

% SQL

SOA; Service Oriented Architecture; Kiến trúc hướng dịch vụ

SOAP; Simple Object Access Protocol; Một giao thức để truy cập dịch vụ web

SPA; Single Page Application; Kiểu ứng dụng một trang

REST; Representational State Transfer; Một tiêu chuẩn thiết kế các API sử dụng cho các dịch vụ web

URL; Uniform Resource Locator ; Địa chỉ định vị tài nguyên trên Internet

XML; Extensible Markup Language; Ngôn ngữ đánh dấu mở rộng

% TCT ; TCT ;

Người nộp thuế ; NNT ;

Mã số thuế ; MST ;

Hóa đơn điện tử ; HĐĐT ;

Cơ quan thuế ; CQT ;

Công nghệ thông tin ; CNTT ;

%%%%%%%%%%%%%%%%%%%%%%%%%%%%%%
% \chapter*{\centering DANH SÁCH CÁC THUẬT NGỮ}

\addcontentsline{toc}{chapter}{DANH SÁCH CÁC THUẬT NGỮ}

% @sau

% @sau

% @sau

% @sau

% @sau

% @sau

% @sau

% @sau

% @sau

% @sau

% @sau

% @sau

% @sau

% @sau

% @sau

% @sau

% @sau

% @sau

% @sau

% @sau

% @sau

% @sau

% @sau

% @sau

% @sau

% @sau

% @sau

% @sau

\begin{table}[h]

\centering

\begin{tabular}{|c|c|c|}

\hline

STT & Tiếng Anh & Tiếng Việt \\

\hline

Dong1 & Dong1 & Cot2 \\

\hline

Dong2 & Dong2 & Cot2 \\

\hline

\end{tabular}

\end{table}

% kiến trúc nguyên khối, kiến trúc nguyên khối

% kiến trúc nguyên khối, kiến trúc nguyên khối

% kiến trúc vi dịch, kiến trúc vi dịch

% kiến trúc vi dịch, kiến trúc vi dịch

% kiến trúc vi dịch, kiến trúc vi dịch

% kiến trúc vi dịch, kiến trúc vi dịch

% thiết kế hướng miền, thiết kế hướng miền

% thiết kế hướng miền, thiết kế hướng miền

1 thiết kế hướng miền

Thiết kế hướng lĩnh vực

2 Domain (không dịch)

3 Abstraction Trừu tượng

4 chuyên gia ngành

% \chapter*{\centering MỞ ĐẦU}
% \addcontentsline{toc}{chapter}{MỞ ĐẦU}
% \section*{Lý do chọn đề tài}
% Trong quá trình hoạt động kinh doanh, doanh nghiệp có nhu cầu chuyển đổi mô hình kinh doanh linh hoạt để có thể tồn tại và phát triển khi thị trường thay đổi. Từ đó, đáp ứng nhu cầu của khách hàng, mang lại ưu thế cạnh tranh so với các đối thủ.

Trong những năm gần đây, việc áp dụng kiến trúc vi dịch vụ ngày càng phổ biến, đem lại nhiều lợi ích như tách các nghiệp vụ kinh doanh thành các dịch vụ nhỏ độc lập, tăng tính linh hoạt và khả năng chống chịu sự cố của hệ thống.

Kiến trúc vi dịch vụ hỗ trợ doanh nghiệp chuyển đổi nhanh chóng để đáp ứng nhu cầu của mô hình kinh doanh và mong đợi của khách hàng. Tuy nhiên, để xây dựng được kiến trúc vi dịch vụ tốt, cần phải tạo ra các dịch vụ nhỏ phù hợp và duy trì tính độc lập. Trong đồ án này, em sử dụng thiết kế hướng miền để phân tích và xây dựng kiến trúc vi dịch vụ.

Theo quy định của Nghị định 123/2020/NĐ - CP, tất cả các doanh nghiệp, tổ chức và hộ kinh doanh đều bắt buộc phải sử dụng hóa điện tử. Vì vậy, nhu cầu sử dụng và xử lý hóa đơn điện tử trở nên rất lớn. Do đó trong đồ án này, em chọn chủ đề \emph{"Sử dụng thiết kế hướng miền xây dựng kiến trúc vi dịch vụ cho bài toán hóa đơn điện tử"}. Chủ đề này là một xu hướng quan trọng trong phát triển phần mềm và mang lại nhiều lợi ích trong việc cải thiện quá trình quản lý hóa đơn điện tử.

% \section*{Đối tượng và phạm vi nghiên cứu}
% \begin{itemize}

\item \textbf{Đối tượng nghiên cứu:} Đối tượng nghiên cứu của đồ án này là phương pháp phát triển phần mềm theo hướng kiến trúc vi dịch vụ có sử dụng thiết kế hướng miền cùng các công nghệ liên quan như: xxxxxxxxxxxxxxxxx, xxxxxxxxxxxxxxxxx, xxxxxxxxxxxxxxxxx, xxxxxxxxxxxxxxxxx, xxxxxxxxxxxxxxxxx, xxxxxxxxxxxxxxxxx, xxxxxxxxxxxxxxxxx, xxxxxxxxxxxxxxxxx,  \dots
  
\item \textbf{Phạm vi nghiên cứu:} Tìm hiểu thiết kế hướng miền xây dựng kiến trúc vi dịch vụ cho bài toán hóa đơn điện tử.

\end{itemize}
% \section*{Tóm tắt nội dung đồ án}
% \input{contents/p001_tom_tat_noi_dung_do_an}

% \chapter{Tổng quan về bài toán hóa đơn điện tử}
% \input{contents/p1_hddt_tong_quan_ve_bai_toan_hoa_don_dien_tu}
% \section{Các khái niệm và căn cứ pháp lý}
% \emph{Theo em tìm hiểu có các khái niệm và căn cứ pháp lý liên quan sau đây:}

\subsection{Hóa đơn}

\emph{Theo quy định tại khoản 1 Điều 3 Nghị định 123/2020/NĐ - CP:}

%%%%%%%%%%%%%%%%%%%%%%%%%%%%%%%%%%%%%!

Hóa đơn là chứng từ kế toán do tổ chức, cá nhân bán hàng hóa, cung cấp dịch vụ lập, ghi nhận thông tin bán hàng hóa, cung cấp dịch vụ. Hóa đơn được thể hiện theo hình thức hóa đơn điện tử hoặc hóa đơn do cơ quan thuế đặt in.

%%%%%%%%%%%%%%%%%%%%%%%%%%%%%%%%%%%%%!

\subsection{Hóa đơn điện tử}

\emph{Theo quy định tại khoản 2 Điều 3 Nghị định 123/2020/NĐ - CP:}

%%%%%%%%%%%%%%%%%%%%%%%%%%%%%%%%%%%%%!

Hóa đơn điện tử là hóa đơn có mã hoặc không có mã của cơ quan thuế được thể hiện ở dạng dữ liệu điện tử do tổ chức, cá nhân bán hàng hóa, cung cấp dịch vụ lập bằng phương tiện điện tử để ghi nhận thông tin bán hàng hóa, cung cấp dịch vụ theo quy định của pháp luật về kế toán, pháp luật về thuế, bao gồm cả trường hợp hóa đơn được khởi tạo từ máy tính tiền có kết nối chuyển dữ liệu điện tử với cơ quan thuế, trong đó:

a. Hóa đơn điện tử có mã của cơ quan thuế là hóa đơn điện tử được cơ quan thuế cấp mã trước khi tổ chức, cá nhân bán hàng hóa, cung cấp dịch vụ gửi cho người mua. Mã của cơ quan thuế trên hóa đơn điện tử bao gồm số giao dịch là một dãy số duy nhất do hệ thống của cơ quan thuế tạo ra và một chuỗi ký tự được cơ quan thuế mã hóa dựa trên thông tin của người bán lập trên hóa đơn.

b. Hóa đơn điện tử không có mã của cơ quan thuế là hóa đơn điện tử do tổ chức bán hàng hóa, cung cấp dịch vụ gửi cho người mua không có mã của cơ quan thuế.

%%%%%%%%%%%%%%%%%%%%%%%%%%%%%%%%%%%%%!

\subsection{Bắt buộc sử dụng hóa đơn điện tử từ 01/07/2022}

\emph{Theo quy định tại khoản 1 Điều 59 Nghị định 123/2020/NĐ - CP:}

%%%%%%%%%%%%%%%%%%%%%%%%%%%%%%%%%%%%%!

Nghị định này có hiệu lực thi hành kể từ ngày 01 tháng 7 năm 2022, khuyến khích cơ quan, tổ chức, cá nhân đáp ứng điều kiện về hạ tầng công nghệ thông tin áp dụng quy định về hóa đơn, chứng từ điện tử của Nghị định này trước ngày 01 tháng 7 năm 2022.

%%%%%%%%%%%%%%%%%%%%%%%%%%%%%%%%%%%%%!

% Chủ đề đồ án

$\Rightarrow$ Theo quy định, tất cả các doanh nghiệp, tổ chức và hộ kinh doanh đều bắt buộc phải chuyển từ sử dụng hóa đơn giấy sang hóa đơn điện tử bắt đầu từ tháng 07/2022. Vì vậy, nhu cầu sử dụng và xử lý hóa đơn điện tử trở nên rất lớn. Do đó ở đồ án này, em chọn chủ đề về quản lý hóa đơn điện tử.

\subsection{Qui định lưu trữ hóa đơn điện tử}

\emph{Theo quy định tại khoản 1 Điều 11 Thông tư 32/2011/TT - BTC:}

%%%%%%%%%%%%%%%%%%%%%%%%%%%%%%%%%%%%%!

Người bán, người mua hàng hoá, dịch vụ sử dụng hóa đơn điện tử để ghi sổ kế toán, lập báo cáo tài chính phải lưu trữ hóa đơn điện tử theo thời hạn quy định của Luật Kế toán. Trường hợp hóa đơn điện tử được khởi tạo từ hệ thống của tổ chức trung gian cung cấp giải pháp hóa đơn điện tử thì tổ chức trung gian này cũng phải thực hiện lưu trữ hóa đơn điện tử theo thời hạn nêu trên.

%%%%%%%%%%%%%%%%%%%%%%%%%%%%%%%%%%%%%!

\emph{Theo quy định tại khoản 5 Điều 41 Luật số 88/2015/QH13:}

%%%%%%%%%%%%%%%%%%%%%%%%%%%%%%%%%%%%%!

1. Tài liệu kế toán phải được lưu trữ theo thời hạn sau đây:

a. Ít nhất là 05 năm đối với tài liệu kế toán dùng cho quản lý, điều hành của đơn vị kế toán, gồm cả chứng từ kế toán không sử dụng trực tiếp để ghi sổ kế toán và lập báo cáo tài chính.

b. Ít nhất là 10 năm đối với chứng từ kế toán sử dụng trực tiếp để ghi sổ kế toán và lập báo cáo tài chính, sổ kế toán và báo cáo tài chính năm, trừ trường hợp pháp luật có quy định khác.

c. Lưu trữ vĩnh viễn đối với tài liệu kế toán có tính sử liệu, có ý nghĩa quan trọng về kinh tế, an ninh, quốc phòng.

%%%%%%%%%%%%%%%%%%%%%%%%%%%%%%%%%%%%%!

% Thời gian lưu trữ

$\Rightarrow$ Như vậy, hóa đơn điện tử sẽ được lưu trữ trên hệ thống hóa đơn điện tử của nhà cung cấp hoặc doanh nghiệp với thời gian lưu trữ ít nhất là 10 năm theo quy định của pháp luật.

\subsection{Một số lợi ích của hóa đơn điện tử}

\emph{Một số lợi ích của hóa đơn điện tử:}

\begin{itemize}

\item Tuân thủ các quy định về thuế và pháp luật.

\item Thể hiện tính minh bạch: bảo vệ quyền lợi của người mua và người bán.

\item Giúp tiết kiệm chi phí in ấn, lưu trữ và bảo quản.

\item Loại bỏ rủi ro cháy, hỏng hoặc mất và dễ dàng sao lưu.

\item Dễ dàng tra cứu, phát hành, quản lý, tạo báo cáo và giảm thủ tục giấy tờ.

\item Giúp theo dõi tình hình tài chính của công ty (doanh thu, chi phí, lợi nhuận).

\end{itemize}



% \section{Yêu cầu nghiệp vụ}
% \input{contents/p1_hddt_yeu_cau_nghiep_vu}

% \subsection{Yêu cầu nghiệp vụ của bài toán phụ}
% \input{contents/p1_hddt_yeu_cau_nghiep_vu_cua_bai_toan_phu}
% \subsubsection{Các chức năng thay đổi so với thực tế trong đồ án này}
% Để đơn giản hóa bài toán, các chức năng trong đồ án này đã thay đổi so với bài toán thực tế như sau:

\begin{itemize}

\item Chữ ký số là QR 2FA

\item xxxxxxxxxx

\item xxxxxxxxxx

\end{itemize}

% % % USB Token hay còn gọi là chữ ký số Token là một thiết bị mà mọi doanh nghiệp, tổ chức hiện nay đều cần phải có để thực hiện khai báo và nộp thuế điện tử, cũng như để giao dịch với khách hàng.

% % % \textbf{Bỏ qua phần ký hiệu hóa đơn}

% % % Vì mục đích của ký hiệu hóa đơn là nhóm 6 ký tự thể hiện thông tin về loại hóa đơn điện tử có mã hoặc không mã, năm lập hóa đơn, loại hóa đơn.

% % % \textbf{Bỏ qua chức năng lập hóa đơn điều chỉnh}

% % % E bỏ qua chức năng lập hóa đơn điều chỉnh và chỉ có chức năng lập hóa đơn thay thế.

% % % \textbf{Bỏ qua chức năng phê duyệt hóa đơn}

% % % \textbf{Bỏ qua định dạng file XML, PDF, HTML, EXCEL}
% \subsubsection{Các chức năng tổng quan của bài toán phụ}
% Các chức năng tổng quan của bài toán phụ bao gồm:

\begin{itemize}

    \item Quản lý tài khoản

          \begin{itemize}

              \item Đăng ký

                    \begin{itemize}

                        \item Để     đăng ký, người nộp thuế cần chuẩn bị:

                              \begin{itemize}

                                  \item Mã số thuế

                                  \item Chữ ký số

                              \end{itemize}

                              %# Code QR

                        \item Người nộp thuế nhập mã số thuế để lấy thông tin thuế bao gồm:

                              \begin{itemize}

                                  \item    Tên người nộp thuế

                                  \item Mã cơ quan thuế
                                  \item Tên cơ quan thuế

                              \end{itemize}



                              %# Code file

                              \begin{vmatrix}

                                  \begin{itemize}

                                      \item Mã số thuế có 10 ký tự đại diện cá nhân, doanh nghiệp hoặc 14 ký tự đại diện chi nhánh của doanh nghiệp với định dạng "Mã số thuế doanh nghiệp-Mã chi nhánh".

                                            \begin{example}

                                                Mã số thuế 10 ký tự: 0123456789

                                                Mã số thuế 14 ký tự: 0123456789-001

                                            \end{example}

                                            %# Code "Mã số thuế phải có độ dài 10 hoặc 14 ký tự và đúng định dạng"

                                      \item Nếu mã số thuế đã tồn tại đăng ký, hệ thống sẽ thông báo: "Mã số thuế đã đăng ký sử dụng hóa đơn điện tử."

                                            %# Code

                                  \end{itemize}
                              \end{vmatrix}

                        \item Tiếp theo, người nộp thuế nhập các thông tin:

                              \begin{itemize}

                                  \item Người liên hệ 

                                  \item Điện thoại liên hệ 

                                  \item Địa chỉ liên hệ 

                                  \item Thư điện tử 

                              \end{itemize} 

                              \begin{vmatrix}

                                  \begin{itemize}

                                      \item Người liên hệ: phải chứa một chuỗi kí tự và không được để trống.

                                            %# Code

                                      \item Điện thoại liên hệ: phải chứa một chuỗi kí tự số và dấu "+" ở đầu (nếu có) và không được để trống.

                                            %# Code

                                      \item Địa chỉ liên hệ: phải chứa một chuỗi kí tự địa chỉ hợp lệ và không được để trống.

                                            %# Code

                                      \item Thư điện tử: phải chứa một chuỗi kí tự có định dạng email và không được để trống.

                                            %# Code

                                  \end{itemize}
                              \end{vmatrix}

                        \item Người nộp thuế chọn hình thức hóa đơn:

                              \begin{itemize}

                                  \item Hóa đơn có mã của cơ quan thuế

                                  \item Hóa đơn không có mã của cơ quan thuế

                              \end{itemize}

                        \item Người nộp thuế chọn loại hóa đơn:

                              \begin{itemize}

                                  \item Hóa đơn điện tử giá trị gia tăng

                                  \item Hóa đơn bán hàng

                                  \item Hóa đơn bán tài sản công

                                  \item Hóa đơn bán hàng dự trữ quốc gia

                                  \item Hóa đơn khác

                              \end{itemize}

                        \item Cuối cùng, người nộp thuế dùng chữ ký số   xác nhận gửi đăng ký với ngày thực hiện là ngày đang đăng ký hóa đơn điện tử.

                              %# Code "Gửi thông tin đăng ký sử dụng hóa đơn điện tử cho cơ quan thuế thành công".

                            $\Rightarrow$  \emph{Sau khi gửi đăng kí, người nộp thuế sẽ nhận được thông báo của cơ quan thuế qua   thư điện tử về việc tiếp nhận và chấp nhận đăng ký.}

                    \end{itemize}

              \item Đăng nhập

                    \begin{itemize}

                        \item Trong thông báo của cơ quan thuế về việc chấp nhận đăng ký, người nộp thuế nhận được thông tin về tài khoản của   người quản trị  bao gồm:


                              \begin{itemize}

                                  \item Tên đăng nhập

                                  \item Mật khẩu

                              \end{itemize}

                        \item      Người nộp thuế    sử dụng  thông tin  để thực hiện đăng nhập tài khoản.
                    \end{itemize}

          \end{itemize}

\end{itemize}



% % \item Đăng xuất (Sign Out)

% % Chức năng để NNT đăng xuất tài khoản.

% % \item Quên mật khẩu (Forgot Password)

% % NNT cung cấp đầy đủ thông tin bao gồm: Tên đăng nhập, Thư điện tử. Sau đó, nhấn "Quên mật khẩu" để khôi phục mật khẩu. cơ quan thuế gửi mật khẩu mới về email của NNT.

% % \item Đổi mật khẩu (Change Password)

% % NNT cung cấp đầy đủ thông tin bao gồm: Mật khẩu cũ, Mật khẩu mới và Nhập lại mật khẩu mới để thực hiện việc thay đổi mật khẩu.

% % \item Thay đổi thông tin (Update Information)

% % Trong quá trình sử dụng hóa đơn điện tử, khi NNT muốn thay đổi đăng ký sử dụng hóa đơn, họ có thể sử dụng chức năng "Thay đổi đăng ký sử dụng hóa đơn điện tử".

% % NNT Nhập thông tin có thể thay đổi, bao gồm: Tên NNT, Người liên hệ, Điện thoại liên hệ, Địa chỉ liên hệ, Thư điện tử.

% % Cuối cùng, NNT gửi đăng ký thay đổi với thông tin "Ngày thực hiện" là ngày NNT đang đăng ký thay đổi hóa đơn điện tử.

% % Sau khi gửi thông tin thay đổi đăng ký, NNT sẽ nhận được thông báo làm việc từ cơ quan thuế qua thư điện tử về việc tiếp nhận và chấp nhận thay đổi đăng ký cho NNT.

% % \end{itemize}

% % \item \underline{QUẢN LÝ HỆ THỐNG}

% % \begin{itemize}

% % \item Quản lý vai trò (Role Management)

% % Người quản trị hệ thống (admin) là một vai trò cố định được phép sử dụng tất cả các chức năng trên Cổng điện tử.

% % Người quản trị hệ thống có thể thực hiện CRUD "Vai trò" với các thông tin bao gồm: "ID", "Tên vai trò" và "Quyền".

% % Các quyền bao gồm:

% % Thay đổi đăng ký sử dụng hóa đơn điện tử

% % Quản lý vai trò

% % Quản lý người dùng

% % Quản lí danh mục

% % Quản lí hóa đơn

% % Tra cứu hóa đơn

% % \item Quản lý người dùng (User Management)

% % Người quản trị hệ thống có thể thực hiện CRUD "Người dùng" với các thông tin bao gồm: "Tên người dùng", "Mật khẩu", "Điện thoại", "Thư điện tử" và "Vai trò".

% % \end{itemize}

% % \item \underline{QUẢN LÝ DANH MỤC}

% % \begin{itemize}

% % \item Quản lý khách hàng (Customer Management)

% % Chức năng này thực hiện CRUD "Khách hàng" có các thông tin: "Mã khách hàng", "Tên khách hàng", "Mã số thuế", "Tên NNT", "Địa chỉ", "SĐT khách hàng", Số tài khoản, Ngân hàng

% % \item Quản lý sản phẩm (Product Management)

% % Chức năng này thực hiện CRUD "Hàng hóa" có các thông tin: "Mã hàng hóa, dịch vụ", "Tên hàng hóa, dịch vụ", "Đơn vị tính", "Đơn giá", "Thuế suất".

% % \end{itemize}

% % \item \underline{QUẢN LÝ HÓA ĐƠN}

% % \begin{itemize}

% % \item Thêm hóa đơn

% % Nhập thông tin người bán: MST người bán, Tên người bán, Địa chỉ người bán, Số điện thoại người bán.

% % Nhập thông tin người mua: Mã khách hàng, Tên khách hàng, Mã số thuế, Địa chỉ khách hàng, SĐT khách hàng.

% % Nhập thông tin hàng hóa, dịch vụ: "Số thứ tự", "Mã hàng hóa, dịch vụ", "Tên hàng hóa, dịch vụ", "Đơn vị tính", "Đơn giá", "Thuế suất" và "Số lượng".

% % Hệ thống tự động tính toán:

% % - Ngày lập hóa đơn sẽ tự động là ngày hiện tại khi người lập tạo hóa đơn mới.

% % - Tổng tiền trước thuế.

% % - Tổng tiền sau thuế.

% % \item Thay thế hóa đơn

% % Chức năng này cho phép thay đổi các thông tin trong hóa đơn gốc.

% % Lưu ý:

% % - Hãy lưu trữ thông tin ID của hóa đơn thay thế trong trạng thái "Bị thay thế" của hóa đơn gốc.

% % - Hãy lưu trữ thông tin ID của hóa đơn gốc trong trạng thái "Thay thế" của hóa đơn thay thế.

% % \item Xóa hóa đơn

% % Chức năng này cho phép xóa hóa đơn và các hóa đơn thay thế liên quan.

% % \end{itemize}

% % \item \underline{TRA CỨU HÓA ĐƠN}

% % Người sử dụng có thể thực hiện tra cứu hóa đơn trên cổng thông tin điện tử theo 2 cách:

% % Cách 1: Tra cứu hóa đơn khi NNT chưa đăng nhập

% % Cách 2: Tra cứu hóa đơn khi NNT đã đăng nhập

% % \begin{itemize}

% % \item Tra cứu khi chưa đăng nhập

% % %!<! - - Tra cứu thông tin hóa đơn - - >

% % Người tra cứu nhập thông tin bao gồm: Mã số thuế người bán, Số hóa đơn, Tổng tiền thuế, Tổng tiền thanh toán, Ngày lập hóa đơn.

% % %!<! - - Kết quả: - - >

% % %!<! - - - Nếu hóa đơn điện tử không hợp lệ, hệ thống sẽ hiển thị thông báo: "Không tồn tại hóa đơn có thông tin trùng khớp với các thông tin tổ chức, cá nhân tìm kiếm”. - - >

% % %!<! - - - Nếu hóa đơn điện tử hợp lệ, hệ thống sẽ hiển thị thông báo: "Tồn tại hóa đơn có thông tin trùng khớp với các thông tin tổ chức, cá nhân tìm kiếm". - - >

% % %!<! - - - Nếu hóa đơn tìm kiếm là hóa đơn thay thế, bị thay thế hệ thống sẽ hiển thị thông tin bổ sung về hóa đơn liên quan: "Hóa đơn này là hóa đơn thay thế cho hóa đơn có ID: {{ID}}" hoặc "Hóa đơn này là hóa đơn bị thay thế của hóa đơn có ID: {{ID}}". - - >

% % %!<! - - Tra cứu thông tin "Mã số thuế" - - >

% % Người tra cứu nhập thông tin bao gồm: Mã số thuế.

% % %!<! - - Kết quả: - - >

% % %!<! - - - Nếu đã đăng kí, hệ thống sẽ hiển thị thông báo: “MST 0107001729 đã đăng ký sử dụng hóa đơn điện tử theo Nghị định 123/2020/NĐ - CP". - - >

% % %!<! - - - Nếu NNT chưa đăng kí hoặc đã đăng kí nhưng cơ quan thuế có thông báo về việc không được chấp nhận đăng kí sử dụng hóa đơn điện tử, hệ thống sẽ hiển thị thông báo: “MST 0107001728 chưa sử dụng hóa đơn điện tử theo Nghị định 123/2020/NĐ - CP". - - >

% % \item Tra cứu khi đã đăng nhập

% % Cổng điện tử hỗ trợ tra cứu 2 loại hóa đơn là hóa đơn bán ra và hóa đơn mua vào.

% % Người tra cứu nhập thông tin tra cứu bao gồm: Mã số thuế người bán, Ngày lập hóa đơn và Số hóa đơn.

% % Cổng điện tử hỗ trợ các chức năng sau: Xem thông tin hóa đơn, In hóa đơn và Xuất hóa đơn (định dạng Excel, XML, PDF).

% % \end{itemize}

% % \item \underline{GỬI PHẢN HỒI QUA THƯ ĐIỆN TỬ}

% % \begin{itemize}

% % \item Gửi thông tin của TCT đến NNT

% % %!<! - - GỬI PHẢN HỒI QUA THƯ ĐIỆN TỬ - - >

% % %!<! - - - Gửi thông tin làm việc của TCT cho yêu cầu của NNT - - >

% % %!<! - - $ NNT nhận được thư điện tử của cơ quan thuế thông báo tiếp nhận tờ khai đăng ký - - >

% % %!<! - - $ NNT nhận được thư điện tử của cơ quan thuế chấp nhận/không chấp nhận đăng ký sử dụng HĐĐT - - >

% % %!<! - - $ NNT nhận được Thông báo tài khoản sử dụng tra cứu HĐĐT trên cổng thông tin điện tử của TCT - - >

% % %!<! - - $ NNT nhận được thư điện tử của cơ quan thuế thông báo tiếp nhận tờ khai đăng ký thay đổi - - >

% % %!<! - - $ NNT nhận được thư điện tử của cơ quan thuế chấp nhận/không chấp nhận đăng ký sử dụng HĐĐT - - >

% % \end{itemize}

% % \end{itemize}

% % $\Rightarrow$ \textbf{Tóm lại, các chức năng tổng quan của bài toán phụ bao gồm:}

% % \begin{itemize}

% % \item \underline{{QUẢN LÝ TÀI KHOẢN}}

% % \begin{itemize}

% % \item Đăng ký (Sign Up)

% % \item Đăng nhập (Sign In)

% % \item Đăng xuất (Sign Out)

% % \item Quên mật khẩu (Forgot Password)

% % \item Đổi mật khẩu (Change Password)

% % \item Thay đổi thông tin (Update Information)

% % \end{itemize}

% % \item \underline{{QUẢN LÝ HỆ THỐNG}}

% % \begin{itemize}

% % \item Quản lý vai trò (Role Management)

% % \item Quản lý người dùng (User Management)

% % \end{itemize}

% % \item \underline{{QUẢN LÝ DANH MỤC}}

% % \begin{itemize}

% % \item Quản lý khách hàng (Customer Management)

% % \item Quản lý sản phẩm (Product Management)

% % \end{itemize}

% % \item \underline{{QUẢN LÝ HÓA ĐƠN}}

% % \begin{itemize}

% % \item Thêm, thay thế, xóa hóa đơn

% % \end{itemize}

% % \item \underline{{TRA CỨU HÓA ĐƠN}}

% % \begin{itemize}

% % \item Tra cứu khi chưa đăng nhập

% % \item Tra cứu khi đã đăng nhập

% % \end{itemize}

% % \item \underline{{GỬI PHẢN HỒI QUA THƯ ĐIỆN TỬ}}

% % \begin{itemize}

% % \item Gửi thông tin của TCT đến NNT

% % \end{itemize}

% % \end{itemize}
% \subsubsection{Chi tiết các chức năng của bài toán phụ}
% \begin{itemize}

\item \underline{{QUẢN LÝ TÀI KHOẢN (Account Management)}}

Quản lý tài khoản là một chức năng phổ biến trong nhiều ứng dụng. Chức năng này đảm bảo tính bảo mật và an toàn trong việc sử dụng tài khoản.

\begin{itemize}

\item Đăng ký (Sign Up)

NNT nhập MST có 10 ký tự cho cá nhân, doanh nghiệp hoặc 14 ký tự cho chi nhánh của doanh nghiệp với định dạng "Mã số thuế doanh nghiệp - Mã chi nhánh".

Ví dụ:

Mã số thuế 10 ký tự: 0123456789

Mã số thuế 14 ký tự: 0123456789 - 001

Hệ thống tự động hiển thị thông tin Đăng ký thuế của NNT bao gồm "Tên của NNT", "Mã cơ quan thuế quản lý" và "Tên cơ quan thuế quản lý".

Tiếp theo, NNT nhập các thông tin hợp lệ: "Người liên hệ", "Điện thoại liên hệ", "Địa chỉ liên hệ", "Thư điện tử".

Cuối cùng, NNT gửi đăng ký với thông tin "Ngày thực hiện" là ngày NNT đang đăng ký hóa đơn điện tử.

Sau khi gửi thông tin đăng kí NNT sẽ nhận được thông báo làm việc của CQT qua gửi thư điện tử về việc tiếp nhận và chấp nhận đăng ký, cùng với tài khoản và mật khẩu cho NNT.

%!<! - - // Nếu mã số thuế không đúng định dạng, hệ thống sẽ thông báo: "Mã số thuế phải có độ dài 10 hoặc 14 ký tự và đúng định dạng". - - >

%!<! - - // Nếu mã số thuế tồn tại, hệ thống kiểm tra xem NNT đã đăng ký sử dụng hóa đơn điện tử khác chưa. Nếu đã tồn tại tờ khai đăng ký, hệ thống thông báo: "Đã tồn tại tờ khai đăng ký sử dụng hóa đơn điện tử khác của NNT đã được cơ quan thuế chấp nhận". - - >

%!<! - - // Người liên hệ: phải chứa một chuỗi kí tự và không được để trống. - - >

%!<! - - // Điện thoại liên hệ: phải chứa một chuỗi kí tự số và dấu " + " ở đầu chuỗi (nếu có) và không được để trống. - - >

%!<! - - // Địa chỉ liên hệ: phải chứa một chuỗi kí tự và không được để trống. - - >

%!<! - - // Thư điện tử: phải chứa một chuỗi kí tự có định dạng email và không được để trống. - - >

%!<! - - // Khi NNT nhấn nút "Ký gửi", hệ thống sẽ hiển thị thông báo hỏi "Xác nhận ký gửi" với hai lựa chọn là "Đồng ý" hoặc "Hủy bỏ". - - >

%!<! - - // Nếu NNT chọn "Đồng ý", hệ thống sẽ thông báo: "Gửi thông tin đăng ký sử dụng hóa đơn điện tử cho cơ quan thuế thành công". - - >

\item Đăng nhập (Sign In)

Sau khi CQT gửi thư điện tử chứa tài khoản và mật khẩu cho NNT, NNT thực hiện nhập đầy đủ thông tin bao gồm: Tên đăng nhập, Mật khẩu để thực hiện việc đăng nhập vào tài khoản.

\item Đăng xuất (Sign Out)

Chức năng để NNT đăng xuất tài khoản.

\item Quên mật khẩu (Forgot Password)

NNT cung cấp đầy đủ thông tin bao gồm: Tên đăng nhập, Thư điện tử. Sau đó, nhấn "Quên mật khẩu" để khôi phục mật khẩu. CQT gửi mật khẩu mới về email của NNT.

\item Đổi mật khẩu (Change Password)

NNT cung cấp đầy đủ thông tin bao gồm: Mật khẩu cũ, Mật khẩu mới và Nhập lại mật khẩu mới để thực hiện việc thay đổi mật khẩu.

\item Thay đổi thông tin (Update Information)

Trong quá trình sử dụng hóa đơn điện tử, khi NNT muốn thay đổi đăng ký sử dụng hóa đơn, họ có thể sử dụng chức năng "Thay đổi đăng ký sử dụng hóa đơn điện tử".

NNT Nhập thông tin có thể thay đổi, bao gồm: Tên NNT, Người liên hệ, Điện thoại liên hệ, Địa chỉ liên hệ, Thư điện tử.

Cuối cùng, NNT gửi đăng ký thay đổi với thông tin "Ngày thực hiện" là ngày NNT đang đăng ký thay đổi hóa đơn điện tử.

Sau khi gửi thông tin thay đổi đăng ký, NNT sẽ nhận được thông báo làm việc từ cơ quan thuế qua thư điện tử về việc tiếp nhận và chấp nhận thay đổi đăng ký cho NNT.

\end{itemize}

\item \underline{{QUẢN LÝ HỆ THỐNG}}

\begin{itemize}

\item Quản lý vai trò (Role Management)

Người quản trị hệ thống (admin) là một vai trò cố định được phép sử dụng tất cả các chức năng trên Cổng điện tử.

Người quản trị hệ thống có thể thực hiện CRUD "Vai trò" với các thông tin bao gồm: "ID", "Tên vai trò" và "Quyền".

Các quyền bao gồm:

Thay đổi đăng ký sử dụng hóa đơn điện tử

Quản lý vai trò

Quản lý người dùng

Quản lí danh mục

Quản lí hóa đơn

Tra cứu hóa đơn

\item Quản lý người dùng (User Management)

Người quản trị hệ thống có thể thực hiện CRUD "Người dùng" với các thông tin bao gồm: "Tên người dùng", "Mật khẩu", "Điện thoại", "Thư điện tử" và "Vai trò".

\end{itemize}

\item \underline{{QUẢN LÝ DANH MỤC}}

\begin{itemize}

\item Quản lý khách hàng (Customer Management)

Chức năng này thực hiện CRUD "Khách hàng" có các thông tin: "Mã khách hàng", "Tên khách hàng", "Mã số thuế", "Tên NNT", "Địa chỉ", "SĐT khách hàng", Số tài khoản, Ngân hàng

\item Quản lý sản phẩm (Product Management)

Chức năng này thực hiện CRUD "Hàng hóa" có các thông tin: "Mã hàng hóa, dịch vụ", "Tên hàng hóa, dịch vụ", "Đơn vị tính", "Đơn giá", "Thuế suất".

\end{itemize}

\item \underline{{QUẢN LÝ HÓA ĐƠN}}

\begin{itemize}

\item Thêm hóa đơn

Nhập thông tin người bán: MST người bán, Tên người bán, Địa chỉ người bán, Số điện thoại người bán.

Nhập thông tin người mua: Mã khách hàng, Tên khách hàng, Mã số thuế, Địa chỉ khách hàng, SĐT khách hàng.

Nhập thông tin hàng hóa, dịch vụ: "Số thứ tự", "Mã hàng hóa, dịch vụ", "Tên hàng hóa, dịch vụ", "Đơn vị tính", "Đơn giá", "Thuế suất" và "Số lượng".

Hệ thống tự động tính toán:

- Ngày lập hóa đơn sẽ tự động là ngày hiện tại khi người lập tạo hóa đơn mới.

- Tổng tiền trước thuế.

- Tổng tiền sau thuế.

\item Thay thế hóa đơn

Chức năng này cho phép thay đổi các thông tin trong hóa đơn gốc.

Lưu ý:

- Hãy lưu trữ thông tin ID của hóa đơn thay thế trong trạng thái "Bị thay thế" của hóa đơn gốc.

- Hãy lưu trữ thông tin ID của hóa đơn gốc trong trạng thái "Thay thế" của hóa đơn thay thế.

\item Xóa hóa đơn

Chức năng này cho phép xóa hóa đơn và các hóa đơn thay thế liên quan.

\end{itemize}

\item \underline{{TRA CỨU HÓA ĐƠN}}

Người sử dụng có thể thực hiện tra cứu hóa đơn trên cổng thông tin điện tử theo 2 cách:

Cách 1: Tra cứu hóa đơn khi NNT chưa đăng nhập

Cách 2: Tra cứu hóa đơn khi NNT đã đăng nhập

\begin{itemize}

\item Tra cứu khi chưa đăng nhập

%!<! - - Tra cứu thông tin hóa đơn - - >

Người tra cứu nhập thông tin bao gồm: Mã số thuế người bán, Số hóa đơn, Tổng tiền thuế, Tổng tiền thanh toán, Ngày lập hóa đơn.

%!<! - - Kết quả: - - >

%!<! - - - Nếu hóa đơn điện tử không hợp lệ, hệ thống sẽ hiển thị thông báo: "Không tồn tại hóa đơn có thông tin trùng khớp với các thông tin tổ chức, cá nhân tìm kiếm”. - - >

%!<! - - - Nếu hóa đơn điện tử hợp lệ, hệ thống sẽ hiển thị thông báo: "Tồn tại hóa đơn có thông tin trùng khớp với các thông tin tổ chức, cá nhân tìm kiếm". - - >

%!<! - - - Nếu hóa đơn tìm kiếm là hóa đơn thay thế, bị thay thế hệ thống sẽ hiển thị thông tin bổ sung về hóa đơn liên quan: "Hóa đơn này là hóa đơn thay thế cho hóa đơn có ID: {{ID}}" hoặc "Hóa đơn này là hóa đơn bị thay thế của hóa đơn có ID: {{ID}}". - - >

%!<! - - Tra cứu thông tin "Mã số thuế" - - >

Người tra cứu nhập thông tin bao gồm: Mã số thuế.

%!<! - - Kết quả: - - >

%!<! - - - Nếu đã đăng kí, hệ thống sẽ hiển thị thông báo: “MST 0107001729 đã đăng ký sử dụng hóa đơn điện tử theo Nghị định 123/2020/NĐ - CP". - - >

%!<! - - - Nếu NNT chưa đăng kí hoặc đã đăng kí nhưng cơ quan thuế có thông báo về việc không được chấp nhận đăng kí sử dụng hóa đơn điện tử, hệ thống sẽ hiển thị thông báo: “MST 0107001728 chưa sử dụng hóa đơn điện tử theo Nghị định 123/2020/NĐ - CP". - - >

\item Tra cứu khi đã đăng nhập

Cổng điện tử hỗ trợ tra cứu 2 loại hóa đơn là hóa đơn bán ra và hóa đơn mua vào.

Người tra cứu nhập thông tin tra cứu bao gồm: Mã số thuế người bán, Ngày lập hóa đơn và Số hóa đơn.

Cổng điện tử hỗ trợ các chức năng sau: Xem thông tin hóa đơn, In hóa đơn và Xuất hóa đơn (định dạng Excel, XML, PDF).

\end{itemize}

\item \underline{{GỬI PHẢN HỒI QUA THƯ ĐIỆN TỬ}}

\begin{itemize}

\item Gửi thông tin của TCT đến NNT

%!<! - - GỬI PHẢN HỒI QUA THƯ ĐIỆN TỬ - - >

%!<! - - - Gửi thông tin làm việc của TCT cho yêu cầu của NNT - - >

%!<! - - $ NNT nhận được thư điện tử của CQT thông báo tiếp nhận tờ khai đăng ký - - >

%!<! - - $ NNT nhận được thư điện tử của CQT chấp nhận/không chấp nhận đăng ký sử dụng HĐĐT - - >

%!<! - - $ NNT nhận được Thông báo tài khoản sử dụng tra cứu HĐĐT trên cổng thông tin điện tử của TCT - - >

%!<! - - $ NNT nhận được thư điện tử của CQT thông báo tiếp nhận tờ khai đăng ký thay đổi - - >

%!<! - - $ NNT nhận được thư điện tử của CQT chấp nhận/không chấp nhận đăng ký sử dụng HĐĐT - - >

\end{itemize}

\end{itemize}

% \subsection{Yêu cầu nghiệp vụ của bài toán chính}
% Yêu cầu nghiệp vụ của bài toán chính bao gồm:

\begin{itemize}

    \item Quản lí thông báo

          Có chức năng tìm kiếm chi tiết, tìm kiếm tất cả, thêm, sửa, xóa đối với  thông báo có các thông tin:

          \begin{itemize}

              \item Mã

              \item Tiêu đề

              \item Nội dung

              \item  Thời gian

          \end{itemize}
    \item Quản lý tài khoản
          Tương tự  như bài toán phụ gồm các chức năng sau:

          \begin{itemize}

              \item Đăng ký

              \item Đăng nhập

              \item Đăng xuất

              \item Quên mật khẩu

              \item Đổi mật khẩu

              \item Thay đổi thông tin

          \end{itemize}
    \item xxxxxxxxxx
    \item xxxxxxxxxx

\end{itemize}



% %!<! - - CẤU HÌNH EMAIL - - >

% Cấu hình bao gồm:

% Địa chỉ email

% Mật khẩu email

% Loại email gửi:

% Xác nhận tài khoản mới

% Quên mật khẩu

% Gửi thông tin hóa đơn cho khách hàng

% %!<! - - QUẢN LÝ DANH MỤC - - >

% Tương tự " TCT Demo" bao gồm:

% Danh mục khách hàng

% Danh mục hàng hóa

% %!<! - - QUẢN LÝ HỆ THỐNG - - >

% Tương tự " TCT Demo" nhưng có thêm quyền "Cấu hình Email".

% %!<! - - QUẢN LÝ HÓA ĐƠN ĐIỆN TỬ - - >

% Tương tự " TCT Demo"

% %!<! - - TRA CỨU HÓA ĐƠN - - >

% Có 3 cách tra cứu:

% Tra cứu 1 hóa đơn theo "Mã hóa đơn"

% Tra cứu tất cả hóa đơn bán ra

% Tra cứu tất cả hóa đơn mua vào

% %!<! - - BÁO CÁO VÀ PHÂN TÍCH HÓA ĐƠN - - >

% Các chức năng bao gồm:

% Số lượng hóa đơn đã sử dụng

% Tổng tiền trước thuế

% Tổng tiền sau thuế

% Tổng số tiền thuế

% Số lượng khách hàng

% Số lượng sản phẩm

% %@ %@ %@ Tự động

% Nghiệp vụ của bài toán chính

% Các chức năng của bài toán chính

% THÔNG BÁO

% CRUD thông báo có (id, tiêu đề, nội dung, thời gian)

% TÀI KHOẢN

% Sử dụng tài khoản của " TCT Demo" với các chức năng tương tự Đăng ký, Đăng nhập, Đăng xuất, Quên mật khẩu, Xem thông tin, Thay đổi thông tin, Đổi mật khẩu

% CẤU HÌNH EMAIL ĐỂ GỬI HÓA ĐƠN CHO KHÁCH HÀNG

% Địa chỉ email

% Mật khẩu email

% CHỨC NĂNG DANH MỤC

% Giống với " TCT Demo" gồm "Danh mục khách hàng" và "Danh mục hàng hóa"

% TRA CỨU HÓA ĐƠN:

% Có 3 cách tra cứu:

% Tra cứu 1 hóa đơn theo "Mã hóa đơn".

% Tra cứu tất cả hóa đơn bán ra.

% Tra cứu tất cả hóa đơn mua vào.

% BÁO CÁO VÀ PHÂN TÍCH HÓA ĐƠN

% Số lượng hóa đơn đã sử dụng

% Tổng trước thuế

% Tổng sau thuế

% Tổng số tiền thuế

% Số lượng khách hàng

% Số lượng sản phẩm

% %!<! - - - - >

% %!<! - - Phân quyền - - >

% %!<! - - Thay đổi - - >

% %!<! - - Lập hóa đơn mới - - >

% %!<! - - Tra cứu - - >

% %!<! - - mail - - >

% \chapter{Giới thiệu về kiến trúc vi dịch vụ}
% \input{contents/p2_msa_gioi_thieu_ve_kien_truc_vi_dich_vu}
% \section{Kiến trúc nguyên khối (Monolithic architecture)}
% Trước khi kiến trúc vi dịch vụ trở nên phổ biến, kiến trúc nguyên khối đã được áp dụng rộng rãi trong kiến trúc phần mềm truyền thống.  \emph{Kiến trúc nguyên khối (Monolithic architecture)} là kiến trúc phần mềm trong đó tất cả các thành phần của dự án được xây dựng thành một đơn vị triển khai duy nhất.

Trong kiến trúc nguyên khối, bất kỳ thay đổi nào đối với một thành phần đều yêu cầu toàn bộ dự án phải được kiểm thử và triển khai lại dẫn đến tốc độ phát triển chậm và thiếu khả năng mở rộng.

Ví dụ: Mô hình MVC (Model - View - Controller) là một trong những mô hình phổ biến  của kiến trúc nguyên khối. Trong mô hình này, ứng dụng được chia thành ba thành phần chính:

\begin{itemize}

\item \textbf{Mô hình (Model):} Đại diện cho dữ liệu và logic xử lý dữ liệu.

\item \textbf{Giao diện (View):} Đại diện cho giao diện người dùng.

\item \textbf{Bộ điều khiển (Controller):} Nhận yêu cầu người dùng thông qua giao diện, sau đó tương tác    làm việc với dữ liệu.

\end{itemize}
% \section{Kiến trúc vi dịch vụ (Microservices architecture)}
% \emph{Kiến trúc vi dịch vụ (Microservices architecture)} chia dự án thành các thành phần nhỏ hơn được gọi là các dịch vụ.

\begin{itemize}

\item Mỗi dịch vụ tập trung vào một khả năng kinh doanh cụ thể.

\item Các dịch vụ độc lập và giao tiếp với nhau thông qua hạ tầng mạng.

\item Trong thực tế, nhiều dự án thường chuyển đổi một cách dần dần từ kiến trúc nguyên khối sang kiến trúc vi dịch vụ.

\end{itemize}

\begin{figure}[H]

\centering

\includegraphics[scale = 0.4]{pictures/_chuyen_doi_tu_kien_truc_nguyen_khoi_sang_kien_truc_vi_dich_vu/main.drawio.png}

\caption{Chuyển đổi từ kiến trúc nguyên khối sang kiến trúc vi dịch vụ}

\end{figure}
% \section{Một số đặc điểm và ưu điểm của kiến trúc vi dịch vụ}
% \input{contents/p2_msa_mot_so_dac_diem_va_uu_diem_cua_kien_truc_vi_dich_vu}
% \section{Một số nhược điểm và thách thức của kiến trúc vi dịch vụ}
% \input{contents/p2_msa_mot_so_nhuoc_diem_va_thach_thuc_cua_kien_truc_vi_dich_vu}

\chapter{Tìm hiểu về thiết kế hướng miền}
% Kiến trúc vi dịch vụ hỗ trợ doanh nghiệp chuyển đổi kinh doanh nhanh và mở rộng hệ thống dễ dàng. Tuy nhiên, để xây dựng được một kiến trúc vi dịch vụ tốt, cần phải tạo ra các dịch vụ nhỏ phù hợp và duy trì tính độc lập. Trong đồ án này, em sử dụng thiết kế hướng miền để phân tích và xây dựng kiến trúc vi dịch vụ. Thiết kế hướng miền giúp xác định và tổ chức các dịch vụ dựa trên việc hiểu rõ về lĩnh vực kinh doanh, từ đó giúp dự án phản ánh đúng các quy trình kinh doanh.
% \section{Đôi nét về thiết kế hướng miền (Domain Driven Design)}
% Thiết kế hướng miền được   \emph{ Eric Evans}     giới thiệu trong cuốn sách \emph{"Domain Driven Design: Tackling Complexity in the Heart of Software"}. \emph{Thiết kế hướng miền (Domain Driven Design)} là một hướng tiếp cận thiết kế phần mềm tập trung vào việc hiểu rõ và mô hình hóa lĩnh vực kinh doanh của một tổ chức. Thiết kế hướng miền nhấn mạnh việc sử dụng lĩnh vực nghiệp vụ kinh doanh để thảo luận và đề xuất giải pháp đáp ứng nhu cầu.

Với nhiều phần mềm được thiết kế không tốt, phần xử lý các công việc không liên quan đến vấn đề nghiệp vụ kinh doanh như truy cập tập tin, hạ tầng mạng, cơ sở dữ liệu, \dots được lập trình trong đối tượng nghiệp vụ kinh doanh. Cách này có ưu điểm giúp tốc độ hoàn thiện phần mềm nhanh. Tuy nhiên, cách này làm dự án bị mất đi tính hướng đối tượng khó thay đổi, mở rộng hệ thống, \dots Thiết kế hướng miền cung cấp một cách để tổ chức mã nguồn và dễ dàng thích ứng với các yêu cầu thay đổi.
% \subsection{Định nghĩa về miền (Domain)}
% 
Hệ thống phần mềm được tạo ra để xử lý công việc trong cuộc sống hiện đại. Việc phát triển hệ thống liên kết chặt chẽ với một số khía cạnh cụ thể trong cuộc sống của chúng ta. Trong thiết kế hướng miền, \emph{miền (Domain)} đề cập đến phạm vi kiến thức và vấn đề cụ thể mà hệ thống xử lý.

\begin{itemize}

\item Về góc độ kinh doanh: Miền đại diện cho một lĩnh vực hoặc ngành mà doanh nghiệp hoạt động.

\item Về góc độ hệ thống: Miền có thể coi là đại diện cho không gian vấn đề của hệ thống.

\end{itemize}

\begin{example} Trong đồ án này, miền được xác định là bài toán giải pháp hóa đơn điện tử. \end{example}
% \subsection{Chuyên gia miền (Domain Expert)}
% Trong thiết kế hướng miền, \emph{chuyên gia miền (Domain Expert)} là người có kiến thức và hiểu biết sâu sắc về vấn đề đang được hệ thống phần mềm giải quyết. Chuyên gia miền thể hiện chính xác vấn đề kinh doanh, đóng vai trò là nguồn thông tin cho nhóm phát triển.

Trong kiến trúc vi dịch vụ, thiết kế hướng miền đảm bảo mỗi dịch vụ được thiết kế phản ánh một phần cụ thể của lĩnh vực kinh doanh. Mỗi dịch vụ được quản lí bởi một nhóm phát triển được hỗ trợ bởi các chuyên gia miền.


% \subsection{Mô hình miền (Domain Models)}
% Để tạo một phần mềm tốt, chúng ta cần phải hiểu rõ về phần mềm đó. \emph{Mô hình miền (Domain Models)} là kiến thức có tổ chức và có cấu trúc về miền phù hợp để giải quyết vấn đề kinh doanh. Mục tiêu của mô hình miền là cung cấp rõ ràng, ngắn gọn và chính xác về miền làm cơ sở để hệ thống giải quyết vấn đề kinh doanh.

\begin{example} Trong đồ án này, mô hình miền của em là các sơ đồ mẫu kỹ thuật ở phần \emph{"Các mẫu kỹ thuật trong thiết kế hướng miền"}. \end{example}
% \subsection{Cốt lõi của thiết kế hướng miền}
% Thiết kế hướng miền cung cấp 2 loại mẫu:

\begin{itemize}

\item \emph{Các mẫu chiến lược (Strategic Patterns):} Phân chia một miền lớn và phức tạp thành các phần nhỏ hơn với ranh giới được xác định rõ ràng. Giúp phân chia một miền lớn hợp lý.

\item \emph{Các mẫu kỹ thuật (Tactical Patterns):} Hiện thực hóa các khái niệm và qui trình thành các thiết kế hệ thống phần mềm. Giúp hệ thống phù hợp với kinh doanh.

\end{itemize}

\begin{figure}[H]

\centering

\includegraphics[scale = 0.5]{pictures/_tong_quan_ve_cot_loi_cua_thiet_ke_huong_mien/main.drawio.png}

\caption{Tổng quan về cốt lõi của thiết kế hướng miền}

\end{figure}



\section{Các mẫu chiến lược trong thiết kế hướng miền}
% Các mẫu chiến lược phân tích nghiệp vụ kinh doanh sau đó đưa ra việc phân chia các thành phần và hiểu mối quan hệ của các thành phần đó. Các mẫu chiến lược giúp xác định các thành phần quan trọng của hệ thống, đảm bảo kiến trúc phần mềm phản ánh đúng các yêu cầu kinh doanh. Từ việc phân chia hệ thống thành các thành phần nhỏ, chúng ta có thể tạo ra hệ thống mở rộng dễ dàng, phát triển linh hoạt theo nhu cầu kinh doanh.

Các mẫu chiến lược bao gồm:

\begin{itemize}

\item Miền phụ (Sub - Domain)

% !trình bày thêm nội dung nhỏ
% !trình bày thêm nội dung nhỏ
% !trình bày thêm nội dung nhỏ
% !trình bày thêm nội dung nhỏ
% !trình bày thêm nội dung nhỏ
% !trình bày thêm nội dung nhỏ
% !trình bày thêm nội dung nhỏ
% !trình bày thêm nội dung nhỏ
% !trình bày thêm nội dung nhỏ
% !trình bày thêm nội dung nhỏ
% !trình bày thêm nội dung nhỏ
% !trình bày thêm nội dung nhỏ

\item Miền phụ (Sub - Domain)

\item Miền phụ (Sub - Domain)

\item Miền phụ (Sub - Domain)

\item Miền phụ (Sub - Domain)

\item Miền phụ (Sub - Domain)

\item Miền phụ (Sub - Domain)

\end{itemize}

\begin{figure}[H]

\centering

\includegraphics[scale = 0.9]{pictures/cac_mau_chien_luoc/temp.png}

\caption{Sơ đồ về các thành phần trong mẫu chiến lược}

\end{figure}

% !<! - - $ Vẽ lại sau khi trình bày xong các mục nhỏ - - >
% \subsection{Miền phụ (Sub - Domain)}
% \input{contents/p3_ddd_mien_phu_sub_domain}
% \subsection{Ngôn ngữ chung (Ubiquitous Language)}
% \input{contents/p3_ddd_ngon_ngu_chung_ubiquitous_language}
% \subsection{Tích hợp liên tục (Continuous Integration)}
% Khi  phân chia miền lớn thành các miền phụ nhỏ hơn, chúng ta cần đảm bảo rằng các miền phụ   luôn ở trạng thái mới  và hoạt động tốt    như kỳ vọng, hạn chế  xảy ra xung đột.   Đáp ứng nhu cầu doanh nghiệp phát triển thay đổi liên tục và nhanh chóng.

\emph{Tích hợp liên tục (Continuous Integration)} là công nghệ tích hợp mã nguồn liên tục, tự động kiểm thử giúp phát hiện và sửa lỗi sớm hơn, giảm thời gian cũng như rủi ro trong quá trình phát triển.

\begin{example} Jenkins là một công cụ tiêu biểu trong công nghệ tích hợp liên tục. \end{example}

\begin{figure}[H]

\centering

\includegraphics[scale = 0.4]{pictures/_vi_du_ve_jenkins_trong_cong_nghe_tich_hop_lien_tuc/main.png}

\caption{Ví dụ về Jenkins trong công nghệ tích hợp liên tục.}

\end{figure} 
\subsection{Bối cảnh bị giới hạn (Bounded Context)}



Để tạo một phần mềm tốt, chúng ta cần phải hiểu rõ về phần mềm đó. \emph{Mô hình miền (Domain Models)} là kiến thức có tổ chức và có cấu trúc về miền phù hợp để giải quyết vấn đề kinh doanh. Mục tiêu của mô hình miền là cung cấp rõ ràng, ngắn gọn và chính xác về miền làm cơ sở để hệ thống giải quyết vấn đề kinh doanh.

\begin{example} Trong đồ án này, mô hình miền của em bao gồm yêu cầu nghiệp vụ, các sơ đồ Use Case và sơ đồ các mẫu kỹ thuật ở phần \ref{section:cac_mau_ky_thuat}. \end{example}
\begin{example} Trong đồ án này, mô hình miền của em bao gồm yêu cầu nghiệp vụ, các sơ đồ Use Case và sơ đồ các mẫu kỹ thuật ở phần \ref{section:cac_mau_ky_thuat}. \end{example}





\section{Các mẫu kỹ thuật trong thiết kế hướng miền} 
\section{Yêu cầu nghiệp vụ} 
\section{Phân tích sơ đồ Use Case} 

\subsection{Mối quan hệ bối cảnh bị giới hạn (Bounded Context Relationships)}
\subsection{Bản đồ bối cảnh (Context Maps)}
\end{document}

% \section{Bối cảnh bị giới hạn (Bounded Context)}

Một miền cần chia đủ nhỏ để phù hợp với một nhóm cụ thể. Để đạt được điều này, chúng ta cần xác định rõ ranh giới giữa các ngữ cảnh. \emph{Bối cảnh bị giới hạn (Bounded Context)} giúp xác định rõ các ranh giới, chia miền thành các phần độc lập để giải quyết sự phức tạp trong mô hình doanh nghiệp. Bối cảnh bị giới hạn tạo ra các mô hình khác nhau cho các lĩnh vực khác nhau của miền. Bối cảnh bị giới hạn thể hiện phạm vi kinh doanh của dịch vụ.

\begin{figure}[H]

\centering

\includegraphics[scale = 1]{pictures/boi_canh_gioi_han/main.png}

\caption{Ví dụ về bối cảnh bị giới hạn trong một ngân hàng}

\end{figure}

\subsubsection{Cách xác định bối cảnh bị giới hạn}

Để có thể xác định được bối cảnh bị giới hạn chúng ta có thể xem xét:

\begin{itemize}

\item Dựa vào việc phân chia các miền phụ.

\item Dựa vào sơ đồ cấu trúc tổ chức các phòng ban của doanh nghiệp.

\item Dựa vào modules của các ứng dụng kiến trúc nguyên khối (nếu việc phân chia tốt).

\item Dựa vào trách nhiệm và hoạt động của chuyên gia miền .

\end{itemize}

\subsubsection{Áp dụng xác định bối cảnh bị giới hạn trong đồ án này}

\subsubsection{Áp dụng xác định bối cảnh bị giới hạn trong đồ án này}

\subsubsection{Áp dụng xác định bối cảnh bị giới hạn trong đồ án này}

\subsubsection{Áp dụng xác định bối cảnh bị giới hạn trong đồ án này}

\subsubsection{Áp dụng xác định bối cảnh bị giới hạn trong đồ án này}

\subsubsection{Áp dụng xác định bối cảnh bị giới hạn trong đồ án này}

\subsubsection{Áp dụng xác định bối cảnh bị giới hạn trong đồ án này}

\subsubsection{Áp dụng xác định bối cảnh bị giới hạn trong đồ án này}

\subsubsection{Áp dụng xác định bối cảnh bị giới hạn trong đồ án này}

\subsubsection{Áp dụng xác định bối cảnh bị giới hạn trong đồ án này}

\subsubsection{Áp dụng xác định bối cảnh bị giới hạn trong đồ án này}

\subsubsection{Áp dụng xác định bối cảnh bị giới hạn trong đồ án này}

%!<! - - Hướng dẫn 5/10 - - >

%!<! - - Hướng dẫn 5/10 - - >

%!<! - - Hướng dẫn 5/10 - - >

%!<! - - Hướng dẫn 5/10 - - >

%!<! - - Hướng dẫn 5/10 - - >

%!<! - - Hướng dẫn 5/10 - - >

%!<! - - Hướng dẫn 5/10 - - >

%!<! - - Hướng dẫn 5/10 - - >

%!<! - - Hướng dẫn 5/10 - - >

% \section{Bản đồ bối cảnh (Context Maps)}

Các bối cảnh bị giới hạn phải độc lập trong bối cảnh riêng và có mô hình miền riêng, nhưng các bối cảnh bị giới hạn cần tương tác, giao tiếp để trao đổi thông tin. Vì vậy các bối cảnh bị giới hạn có thể có mối quan hệ với nhau. Những mối quan hệ này cần được quản lý chặt chẽ để hoạt động độc lập, nhất quán và linh hoạt. Do đó, cần phải ghi lại các mối quan hệ thông qua việc sử dụng bản đồ bối cảnh. \emph{Bản đồ bối cảnh (Context Maps)} là sự thể hiện trực quan của hệ thống, thể hiện các thành phần và mối quan hệ giữa các thành phần.

\begin{figure}[H]

\centering

\includegraphics[scale = 0.4]{pictures/ban_do_boi_canh/main.drawio.png}

\caption{Ví dụ bản đồ bối cảnh trong 1 ngân hàng}

\end{figure}

%! Vẽ lại tiếng Việt

%! Vẽ lại tiếng Việt

%! Vẽ lại tiếng Việt

%! Vẽ lại tiếng Việt

%! Vẽ lại tiếng Việt

%! Vẽ lại tiếng Việt

%! Vẽ lại tiếng Việt

%! Vẽ lại tiếng Việt

%! Vẽ lại tiếng Việt

%! Vẽ lại tiếng Việt

% \section{Các mối quan hệ bối cảnh bị giới hạn}

Có 3 loại mối quan hệ giữa các bối cảnh bị giới hạn là:

\begin{itemize}

\item Mối quan hệ đối xứng (Symmetric Relationship)

\textbf{Mô tả:} Thể hiện sự tương tác 2 chiều giữa 2 bối cảnh bị giới hạn .

\item Mối quan hệ bất đối xứng (Asymmetric Relationship)

\textbf{Mô tả:} Thể hiện sự tương tác 1 chiều giữa 2 các bối cảnh bị giới hạn .

\item Mối quan hệ 1 - nhiều (One to Many Relationship)

\textbf{Mô tả:} Thể hiện sự tương tác 1 chiều giữa 1 bối cảnh bị giới hạn với nhiều bối cảnh bị giới hạn khác.

\end{itemize}

\begin{figure}[H]

\centering

\includegraphics[scale = 0.5]{pictures/cac_moi_quan_he_boi_canh_gioi_han/main.png}

\caption{Các mối quan hệ bối cảnh bị giới hạn}

\end{figure}

%%%%%%%%%%%%%%%%%%%%%%%%%%%%%%%%%%

%%%%%%%%%%%%%%%%%%%%%%%%%%%%%%%%%%

%%%%%%%%%%%%%%%%%%%%%%%%%%%%%%%%%%

%%%%%%%%%%%%%%%%%%%%%%%%%%%%%%%%%%

%%%%%%%%%%%%%%%%%%%%%%%%%%%%%%%%%%

%%%%%%%%%%%%%%%%%%%%%%%%%%%%%%%%%%

%%%%%%%%%%%%%%%%%%%%%%%%%%%%%%%%%%

% \subsection{Mối quan hệ đối xứng (Symmetric Relationship)}

% \subsubsection{Mô hình riêng biệt (Separate Ways)}

Mô hình riêng biệt (Separate Ways) khi các bối cảnh bị giới hạn có quan hệ riêng biệt, không có sự phụ thuộc. Vì vậy, các bối cảnh bị giới hạn này có ngôn ngữ, mô hình, mục đích độc lập và thực thi riêng biệt. Các nhóm phát triển không phải cộng tác hay phối hợp với nhau từ đó đem lại lợi ích dễ dàng bảo trì và mở rộng hệ thống.

\begin{example} Trong miền vấn đề ngân hàng, thẻ tín dụng và khoản vay mua nhà không có mối quan hệ.

\begin{figure}[H]

\centering

\includegraphics[scale = 0.5]{pictures/mo_hinh_rieng_biet_separate_ways/main.drawio.png}

\caption{Ví dụ mô hình riêng biệt (Separate Ways)}

\end{figure}

\end{example}

% \subsubsection{Mô hình hạt nhân chung (Shared Kernel)}

Trong thực tế, nhiều bối cảnh bị giới hạn phụ thuộc lẫn nhau. Mô hình hợp tác (Partnership) tạo điều kiện cho việc giao tiếp và cộng tác giữa các bối cảnh bị giới hạn phụ thuộc. Tuy nhiên, sự phụ thuộc này dẫn đến mức độ kết hợp cao giữa các nhóm và bối cảnh bị giới hạn, dẫn tới mất đi tính độc lập.

\emph{Lưu ý: Mô hình hợp tác không phải là mô hình của các mẫu chiến lược trong thiết kế huớng miền.}

Để giải quyết vấn đề bối cảnh bị giới hạn phụ thuộc lẫn nhau chúng ta có mô hình hạt nhân chung. Mô hình hạt nhân chung (Shared Kernel) cho phép các bối cảnh bị giới hạn có phần chia sẻ chung và có ranh giới phân định rõ ràng. Từ đó, tách việc quản lí các mô hình hạt nhân chung này một cách độc lập với phần còn lại của bối cảnh bị giới hạn . Khi cần thay đổi mà không phải của mô hình hạt nhân chung thì nhóm sẽ hoạt động độc lập. Thông thường, mô hình hạt nhân chung được hiện thực hóa bằng các thư viện chung. Tuy nhiên, chỉ sử dụng mô hình hạt nhân chung nếu quan hệ của các bối cảnh bị giới hạn nhỏ và ổn định để tránh quan hệ phức tạp và ràng buộc chặt chẽ.

% Vẽ lại bản đồ tiếng Việt

% Vẽ lại bản đồ tiếng Việt

% Vẽ lại bản đồ tiếng Việt

% Vẽ lại bản đồ tiếng Việt

% Vẽ lại bản đồ tiếng Việt

% Vẽ lại bản đồ tiếng Việt

% Vẽ lại bản đồ tiếng Việt

% Vẽ lại bản đồ tiếng Việt

% Từ bản đồ lấy vi dụ cho các mô hình

% Từ bản đồ lấy vi dụ cho các mô hình

% Từ bản đồ lấy vi dụ cho các mô hình

% Từ bản đồ lấy vi dụ cho các mô hình

% Từ bản đồ lấy vi dụ cho các mô hình

% Từ bản đồ lấy vi dụ cho các mô hình

% Từ bản đồ lấy vi dụ cho các mô hình

% Từ bản đồ lấy vi dụ cho các mô hình

% Từ bản đồ lấy vi dụ cho các mô hình

% Từ bản đồ lấy vi dụ cho các mô hình

\begin{example} Trong miền vấn đề ngân hàng, thẻ tín dụng và khoản vay mua nhà không có mối quan hệ.

\begin{figure}[H]

\centering

\includegraphics[scale = 0.5]{pictures/mo_hinh_rieng_biet_separate_ways/main.drawio.png}

\caption{Ví dụ mô hình riêng biệt (Separate Ways)}

\end{figure}

\end{example}

% %! $VD: hình giao như 2 tập hợp - - >

% \subsection{Mối quan hệ bất đối xứng (Asymmetric Relationship)}

Trong mối quan hệ bất đối xứng, một bối cảnh bị giới hạn có sự phụ thuộc vào một bối cảnh bị giới hạn khác. Mối quan hệ này được mô tả bằng cách gán vai trò cho bối cảnh bị giới hạn :

\begin{itemize}

\item \textbf{Bối cảnh bị giới hạn thượng nguồn (Upstream):}

\begin{itemize}

\item Bối cảnh bị giới hạn cung cấp cho bối cảnh bị giới hạn khác.

\item Ký hiệu: U

\end{itemize}

\item \textbf{Bối cảnh bị giới hạn hạ lưu (Downstream):}

\begin{itemize}

\item Bối cảnh bị giới hạn phụ thuộc vào bối cảnh bị giới hạn khác.

\item Ký hiệu: D

\end{itemize}

\end{itemize}

\begin{example} Mối quan hệ bất đối xứng giữa bối cảnh bị giới hạn A và bối cảnh bị giới hạn B.

\begin{itemize}

\item Bối cảnh bị giới hạn A ràng buộc với bối cảnh bị giới hạn B

\item Bối cảnh bị giới hạn A đóng vai trò là bối cảnh bị giới hạn hạ lưu (Downstream)

\item Bối cảnh bị giới hạn B đóng vai trò là bối cảnh bị giới hạn thượng nguồn (Upstream)

\item Bối cảnh bị giới hạn A có kiến thức về các mô hình trong bối cảnh bị giới hạn B

\item Bối cảnh bị giới hạn B không có bất kỳ kiến thức nào về mô hình trong bối cảnh bị giới hạn A

\end{itemize}

\begin{figure}[H]

\centering

\includegraphics[scale = 0.5]{pictures/moi_quan_he_bat_doi_xung/main.drawio.png}

\caption{Ví dụ mối quan hệ bất đối xứng}

\end{figure}

\end{example}

% \subsubsection{Mô hình khách hàng - nhà cung cấp (Customer - Supplier)}

Mô hình khách hàng - nhà cung cấp (Customer - Supplier) được thể hiện rằng bối cảnh bị giới hạn thượng nguồn đáp ứng nhu cầu của bối cảnh bị giới hạn hạ lưu.

Khi đó:

\begin{itemize}

\item Bối cảnh bị giới hạn thượng nguồn được gọi là nhà cung cấp.

\item Bối cảnh bị giới hạn hạ lưu được gọi là khách hàng.

\end{itemize}

Trong thực tế, nhóm phát triển nhà cung cấp luôn tham khảo ý kiến của nhóm phát triển khách hàng và có bộ kiểm thử để đảm bảo rằng dịch vụ của nhà cung cấp đáp ứng được yêu cầu của khách hàng.

% \subsubsection{Mô hình tuân thủ (Conformist)}

Trong mô hình khách hàng - nhà cung cấp, nếu nhà cung cấp thực hiện tốt yêu cầu thì khách hàng cần tuân thủ chặt chẽ. Mô hình tuân thủ (Conformist) là một mối quan hệ trong đó bối cảnh bị giới hạn hạ lưu áp dụng mô hình, ngôn ngữ chung và các khái niệm của bối cảnh bị giới hạn thượng nguồn.

Trong mô hình tuân thủ bối cảnh bị giới hạn hạ lưu được ký hiệu là CF.

%! $VD: - - >

%! $VD: A - CF - U - B - - >

%! $VD: A - users(id, name) - B cũng users(id, name) - - >

% Vẽ lại bản đồ tiếng Việt

% Vẽ lại bản đồ tiếng Việt

% Vẽ lại bản đồ tiếng Việt

% Vẽ lại bản đồ tiếng Việt

% Vẽ lại bản đồ tiếng Việt

% Vẽ lại bản đồ tiếng Việt

% Vẽ lại bản đồ tiếng Việt

% Vẽ lại bản đồ tiếng Việt

% Từ bản đồ lấy vi dụ cho các mô hình

% Từ bản đồ lấy vi dụ cho các mô hình

% Từ bản đồ lấy vi dụ cho các mô hình

% Từ bản đồ lấy vi dụ cho các mô hình

% Từ bản đồ lấy vi dụ cho các mô hình

% Từ bản đồ lấy vi dụ cho các mô hình

% Từ bản đồ lấy vi dụ cho các mô hình

% Từ bản đồ lấy vi dụ cho các mô hình

% Từ bản đồ lấy vi dụ cho các mô hình

% Từ bản đồ lấy vi dụ cho các mô hình

\begin{example} Trong miền vấn đề ngân hàng, thẻ tín dụng và khoản vay mua nhà không có mối quan hệ.

\begin{figure}[H]

\centering

\includegraphics[scale = 0.5]{pictures/mo_hinh_rieng_biet_separate_ways/main.drawio.png}

\caption{Ví dụ mô hình riêng biệt (Separate Ways)}

\end{figure}

\end{example}

% \subsubsection{Mô hình chống đổ vỡ (Anti Corruption Layer)}

Trong mô hình khách hàng - nhà cung cấp, nếu nhà cung cấp có thể thay đổi linh hoạt không đảm bảo đáp ứng nhu cầu của khách hàng thì khách hàng cần có giải pháp xử lí. Mô hình chống đổ vỡ (Anti Corruption Layer) là một mối quan hệ trong đó bối cảnh bị giới hạn hạ lưu sử dụng một lớp để dịch giữa ngôn ngữ của nó và ngôn ngữ của bối cảnh bị giới hạn thượng nguồn.

Trong mô hình chống đổ vỡ, mỗi bối cảnh bị giới hạn có mô hình riêng biệt và lớp chống đổ vỡ cần kiến thức về mô hình hạ lưu và thượng nguồn để bảo vệ hạ lưu và duy trì tính toàn vẹn.

%@ Façade

%@ Adapter

Trong mô hình chống đổ vỡ bối cảnh bị giới hạn hạ lưu được ký hiệu là ACL.

% Vẽ lại bản đồ tiếng Việt

% Vẽ lại bản đồ tiếng Việt

% Vẽ lại bản đồ tiếng Việt

% Vẽ lại bản đồ tiếng Việt

% Vẽ lại bản đồ tiếng Việt

% Vẽ lại bản đồ tiếng Việt

% Vẽ lại bản đồ tiếng Việt

% Vẽ lại bản đồ tiếng Việt

% Từ bản đồ lấy vi dụ cho các mô hình

% Từ bản đồ lấy vi dụ cho các mô hình

% Từ bản đồ lấy vi dụ cho các mô hình

% Từ bản đồ lấy vi dụ cho các mô hình

% Từ bản đồ lấy vi dụ cho các mô hình

% Từ bản đồ lấy vi dụ cho các mô hình

% Từ bản đồ lấy vi dụ cho các mô hình

% Từ bản đồ lấy vi dụ cho các mô hình

% Từ bản đồ lấy vi dụ cho các mô hình

% Từ bản đồ lấy vi dụ cho các mô hình

\begin{example} Trong miền vấn đề ngân hàng, thẻ tín dụng và khoản vay mua nhà không có mối quan hệ.

\begin{figure}[H]

\centering

\includegraphics[scale = 0.5]{pictures/mo_hinh_rieng_biet_separate_ways/main.drawio.png}

\caption{Ví dụ mô hình riêng biệt (Separate Ways)}

\end{figure}

\end{example}

% \subsection{Mối quan hệ 1 - nhiều (One to Many Relationship)}

% \subsubsection{Dịch vụ máy chủ mở (Open Host Service)}

Dịch vụ máy chủ mở (Open Host Service) là nhà cung cấp trong mô hình khách hàng - nhà cung cấp, dịch vụ máy chủ mở hiển thị một API công khai cho các bối cảnh bị giới hạn khác sử dụng chức năng của nhà cung cấp.

Trong bản đồ bối cảnh, dịch vụ máy chủ mở được ký hiệu là OHS.

% Vẽ lại bản đồ tiếng Việt

% Vẽ lại bản đồ tiếng Việt

% Vẽ lại bản đồ tiếng Việt

% Vẽ lại bản đồ tiếng Việt

% Vẽ lại bản đồ tiếng Việt

% Vẽ lại bản đồ tiếng Việt

% Vẽ lại bản đồ tiếng Việt

% Vẽ lại bản đồ tiếng Việt

% Từ bản đồ lấy vi dụ cho các mô hình

% Từ bản đồ lấy vi dụ cho các mô hình

% Từ bản đồ lấy vi dụ cho các mô hình

% Từ bản đồ lấy vi dụ cho các mô hình

% Từ bản đồ lấy vi dụ cho các mô hình

% Từ bản đồ lấy vi dụ cho các mô hình

% Từ bản đồ lấy vi dụ cho các mô hình

% Từ bản đồ lấy vi dụ cho các mô hình

% Từ bản đồ lấy vi dụ cho các mô hình

% Từ bản đồ lấy vi dụ cho các mô hình

\begin{example} Trong miền vấn đề ngân hàng, thẻ tín dụng và khoản vay mua nhà không có mối quan hệ.

\begin{figure}[H]

\centering

\includegraphics[scale = 0.5]{pictures/mo_hinh_rieng_biet_separate_ways/main.drawio.png}

\caption{Ví dụ mô hình riêng biệt (Separate Ways)}

\end{figure}

\end{example}

% \subsubsection{Ngôn ngữ được xuất bản (Published Language)}

Khi ngôn ngữ chung ở dịch vụ máy chủ mở được các nhóm phát triển trong bối cảnh bị giới hạn hạ lưu chấp nhận. Ngôn ngữ chung này được gọi là ngôn ngữ được xuất bản (Published Language). Ngôn ngữ được xuất bản có lợi ích là tính thống nhất trong hệ thống tuy nhiên cần phân tích kĩ vì nó có thể tạo ra sự nhầm lẫn trong bối cảnh bị giới hạn hạ lưu nào đó.

Trong bản đồ bối cảnh, ngôn ngữ được xuất bản kết hợp dịch vụ máy chủ mở được ký hiệu là OHS|PL.

% Vẽ lại bản đồ tiếng Việt

% Vẽ lại bản đồ tiếng Việt

% Vẽ lại bản đồ tiếng Việt

% Vẽ lại bản đồ tiếng Việt

% Vẽ lại bản đồ tiếng Việt

% Vẽ lại bản đồ tiếng Việt

% Vẽ lại bản đồ tiếng Việt

% Vẽ lại bản đồ tiếng Việt

% Từ bản đồ lấy vi dụ cho các mô hình

% Từ bản đồ lấy vi dụ cho các mô hình

% Từ bản đồ lấy vi dụ cho các mô hình

% Từ bản đồ lấy vi dụ cho các mô hình

% Từ bản đồ lấy vi dụ cho các mô hình

% Từ bản đồ lấy vi dụ cho các mô hình

% Từ bản đồ lấy vi dụ cho các mô hình

% Từ bản đồ lấy vi dụ cho các mô hình

% Từ bản đồ lấy vi dụ cho các mô hình

% Từ bản đồ lấy vi dụ cho các mô hình

\begin{example} Trong miền vấn đề ngân hàng, thẻ tín dụng và khoản vay mua nhà không có mối quan hệ.

\begin{figure}[H]

\centering

\includegraphics[scale = 0.5]{pictures/mo_hinh_rieng_biet_separate_ways/main.drawio.png}

\caption{Ví dụ mô hình riêng biệt (Separate Ways)}

\end{figure}

\end{example}

%%%%%%%%%%%%%%%%%%%%%%%%%%%%%%%%%%

%%%%%%%%%%%%%%%%%%%%%%%%%%%%%%%%%%

%%%%%%%%%%%%%%%%%%%%%%%%%%%%%%%%%%

%%%%%%%%%%%%%%%%%%%%%%%%%%%%%%%%%%

%%%%%%%%%%%%%%%%%%%%%%%%%%%%%%%%%%

%%%%%%%%%%%%%%%%%%%%%%%%%%%%%%%%%%

%%%%%%%%%%%%%%%%%%%%%%%%%%%%%%%%%%

%%%%%%%%%%%%%%%%%%%%%%%%%%%%%%%%%%

% \section{Áp dụng về các mối quan hệ bối cảnh bị giới hạn}

%! Hướng dẫn 6/6

%%%%%%%%%%%%%%%%%%%%%%%%%%%%%%%%%%

%%%%%%%%%%%%%%%%%%%%%%%%%%%%%%%%%%

%%%%%%%%%%%%%%%%%%%%%%%%%%%%%%%%%%

%%%%%%%%%%%%%%%%%%%%%%%%%%%%%%%%%%

%%%%%%%%%%%%%%%%%%%%%%%%%%%%%%%%%%

%%%%%%%%%%%%%%%%%%%%%%%%%%%%%%%%%%

%%%%%%%%%%%%%%%%%%%%%%%%%%%%%%%%%%

%%%%%%%%%%%%%%%%%%%%%%%%%%%%%%%%%%

\section{Các mẫu kỹ thuật trong thiết kế hướng miền}
%@%%%%%%%%%%%%%%%%%%%%%%%%%%%% now()
\end{document}
%@%%%%%%%%%%%%%%%%%%%%%%%%%%%% now()
% \section{Đối tượng miền (Domain Object)}
% \subsection{Đối tượng thực thể (Entities Objects)}
% % \subsection{Đối tượng thực thể (Entities Objects)}
% \subsection{Đối tượng giá trị (Value Objects)}
% \subsection{Miền dịch vụ (Service Domain)}

% \subsubsection{Phân loại các miền phụ}
% \subsubsection{Cách xác định các miền phụ}
% % \subsubsection{Mô tả cách xác định các miền phụ}
% % \section{Mô hình miền (Domain Models)}
% % \section{Bối cảnh bị giới hạn (Bounded Context)}
% \subsubsection{Cách xác định bối cảnh bị giới hạn}
% \subsubsection{Áp dụng xác định bối cảnh bị giới hạn trong đồ án này}
% % \section{Ngôn ngữ chung (Ubiquitous Language)}
% \subsubsection{Một số đặc điểm của ngôn ngữ chung}
% % \section{Bản đồ bối cảnh (Context Maps)}
% % \section{Các mối quan hệ bối cảnh bị giới hạn}
% % \subsection{Mối quan hệ đối xứng (Symmetric Relationship)}
% % \subsubsection{Mô hình riêng biệt (Separate Ways)}
% % \subsubsection{Mô hình hạt nhân chung (Shared Kernel)}
% % \subsection{Mối quan hệ bất đối xứng (Asymmetric Relationship)}
% % \subsubsection{Mô hình khách hàng - nhà cung cấp (Customer - Supplier)}
% % \subsubsection{Mô hình tuân thủ (Conformist)}
% % \subsubsection{Mô hình chống đổ vỡ (Anti Corruption Layer)}
% % \subsection{Mối quan hệ 1 - nhiều (One to Many Relationship)}
% % \subsubsection{Dịch vụ máy chủ mở (Open Host Service)}
% % \subsubsection{Ngôn ngữ được xuất bản (Published Language)}
% % \section{Áp dụng về các mối quan hệ bối cảnh bị giới hạn}
% \section{Đối tượng miền (Domain Object)}
% \subsection{Đối tượng thực thể (Entities Objects)}
% % \subsection{Đối tượng thực thể (Entities Objects)}
% \subsection{Đối tượng giá trị (Value Objects)}
% \subsection{Miền dịch vụ (Service Domain)}
% % \subsubsection{xxxxxxx}
% % \subsubsection{xxxxxxx}
\end{document}
%#%%%%%%%%%%%%%%%%%%%%%%%%%%%%
%%%%%%%%%%%%%%%%%%%%%%%%%%%%%%
%%%%%%%%%%%%%%%%%%%%%%%%%%%%%%
%%%%%%%%%%%%%%%%%%%%%%%%%%%%%%
%%%%%%%%%%%%%%%%%%%%%%%%%%%%%%
%%%%%%%%%%%%%%%%%%%%%%%%%%%%%%
%%%%%%%%%%%%%%%%%%%%%%%%%%%%%%
%%%%%%%%%%%%%%%%%%%%%%%%%%%%%%
%%%%%%%%%%%%%%%%%%%%%%%%%%%%%%
%%%%%%%%%%%%%%%%%%%%%%%%%%%%%%
%# Một số công nghệ trong kiến trúc vi dịch vụ
%# Một số công nghệ trong kiến trúc vi dịch vụ
%# Một số công nghệ trong kiến trúc vi dịch vụ
%# Một số công nghệ trong kiến trúc vi dịch vụ
%# Một số công nghệ trong kiến trúc vi dịch vụ
%# Một số công nghệ trong kiến trúc vi dịch vụ
%# Một số công nghệ trong kiến trúc vi dịch vụ
%# Một số công nghệ trong kiến trúc vi dịch vụ
%# Một số công nghệ trong kiến trúc vi dịch vụ
%# Một số công nghệ trong kiến trúc vi dịch vụ
%# Một số công nghệ trong kiến trúc vi dịch vụ
%# Một số công nghệ trong kiến trúc vi dịch vụ
%# Một số công nghệ trong kiến trúc vi dịch vụ
%# Một số công nghệ trong kiến trúc vi dịch vụ
%# Một số công nghệ trong kiến trúc vi dịch vụ
%# Một số công nghệ trong kiến trúc vi dịch vụ
%# Một số công nghệ trong kiến trúc vi dịch vụ
%# Một số công nghệ trong kiến trúc vi dịch vụ
%# Một số công nghệ trong kiến trúc vi dịch vụ
%# Một số công nghệ trong kiến trúc vi dịch vụ
%# Một số công nghệ trong kiến trúc vi dịch vụ
%# Một số công nghệ trong kiến trúc vi dịch vụ
%# Một số công nghệ trong kiến trúc vi dịch vụ
%# Một số công nghệ trong kiến trúc vi dịch vụ
%# Một số công nghệ trong kiến trúc vi dịch vụ
%# Một số công nghệ trong kiến trúc vi dịch vụ
%# Một số công nghệ trong kiến trúc vi dịch vụ
%# Một số công nghệ trong kiến trúc vi dịch vụ
%# Một số công nghệ trong kiến trúc vi dịch vụ
%#%%%%%%%%%%%%%%%%%%%%%%%%%%%%
\chapter{Một số công nghệ trong kiến trúc vi dịch vụ}
% % \section{xxxxxxxxxxxxxxxxxx}

% % phải có CQRS (Phân chia trách nhiệm truy vấn lệnh)

% CQRS là một mẫu kiến trúc riêng biệt có thể được sử dụng kết hợp với thiết kế hướng miền để đạt được những lợi ích nhất định, chẳng hạn như cải thiện hiệu suất và khả năng mở rộng. Tuy nhiên, nó không phải là một yêu cầu để triển khai thiết kế hướng miền.

% % phải có event

% Cách tiếp cận này nhấn mạnh tính mô - đun, tính linh hoạt và khả năng phục hồi, cho phép các nhóm làm việc đồng thời trên các phần khác nhau của hệ thống và cho phép phát hành nhanh hơn và thường xuyên hơn. Các vi dịch vụ thường dựa vào các giao thức truyền thông nhẹ, chẳng hạn như REST và thường được triển khai bằng các công nghệ chứa trong bộ chứa như Docker và Kubernetes.

% \subsubsection{DevOps Ứng dụng, áp dụng, liên quan, ....}

% \subsubsection{Github}

% \subsubsection{CI/CD}

% \subsubsection{Docker}

% \subsubsection{Kubernetes}

% dícovery

% % api gateway

% Repository độc lập miền và lưu trữ sql (dễ tuhaajn tiện Unit testing and Mocking)

% Repository trong ORM

% <!--https: //images.viblo.asia/fd4b10a0-f1b1-4ed1-9bd1-578c871820ae.png-->

%, gprc rabitmq đồng bộ hay k, ít hay nhiều như pub sub

% # 5. Service Mesh, CICD, microfe, API gateway, log xử lí lỗi,

% <!---->

% <!---->

% <!--Thay thế = NULL-->

% <!--Bị thay thế = NULL-->

% <!--quy trình tương tự như lập mới hóa đơn giá trị gia tăng.-->

% <!---->

% <!--@Chú ý ở đồ án này:-->

% <!--Sử dụng hàm ngẫu nhiên (tỉ lệ 10%) cho trường hợp "Mã số thuế không tồn tại."-->

% <!--Sử dụng hàm ngẫu nhiên tạo tên cho Tên NNT vì em không có thông tin đăng ký thực tế của NNT.-->

% <!--Sử dụng hàm ngẫu nhiên trong bảng CSDL cho "Mã cơ quan thuế quản lý" và "Tên cơ quan thuế quản lý"-->

% Bảng CSDL này được em thu thập dữ liệu từ trang web CƠ SỞ DỮU DANH MỤC DÙNG CHUNG (https: //dmdc.mof.gov.vn/khai-thac-pb/co-quan-thue)

% <!--!Mã thuế số-chi nhánh-->

% <!--Mã captcha không đúng.-->

% <!--0107001729-->

% dấu chấm cuối câu .

% email=>Thư điện tử

% Viết tắt NNT...

% <!--Validtae-->

% Điều kiện

% <!---->

% Chỉ dùng 1 loại hóa đơn vì em thấy tương tự.

% Loại hóa đơn: + Hóa đơn giá trị gia tăng + Hóa đơn bán hàng + Hóa đơn bán tài sản công + Hóa đơn bán hàng dự trữ quốc gia + Hóa đơn khác + Chứng từ điện tử được sử dụng và quản lý như hóa đơn

% <!--Nghiệp vụ của bài toán chính-->

% Video Viettel

% <!--@Chú ý ở đồ án này:-->

% Mã giao dịch điện tử = Mã số thuế + Thời gian đăng kí

% Sử dụng hàm ngẫu nhiên (tỉ lệ 10%) cho trường hợp từ chối.

% <!--Phân tích và thiết kế-->

% Xác định các tính năng cần thiết và các yêu cầu kỹ thuật tạo ra một thiết kế hệ thống hoặc kiến trúc đáp ứng.

% <!---->

% <!--Các công nghệ phổ biến trong kiến trúc vi dịch vụ-->

% Docker container.....

% Docker container.....

% Docker container.....

% Docker container.....

% [](0.9.KetLuan_TongKet.md)

% [](_.TaiLieuThamKhao.md)

% <!--RxJS-->

% https: //www.youtube.com/watch? v=6jSk_J7RA24

% https: //www.youtube.com/watch? v=Jc-lGeDuphg

% https: //www.youtube.com/watch? v=UXHzxX4png0

% https: //www.youtube.com/watch? v=glZs4QFfwbc

% # 6. Container và Container Orchestration

% Docker and Kubernetes (often abbreviated as K8s) are two powerful technologies commonly used in the world of container orchestration and deployment. Let's briefly explore each of them:

% 1. **Docker: **

% - **Containerization Technology: ** Docker is a platform that enables developers to automate the deployment of applications inside lightweight, portable containers. Containers encapsulate an application and its dependencies, ensuring consistency across different environments.

% - **Docker Image: ** A Docker image is a lightweight, standalone, executable package that includes everything needed to run a piece of software, including the code, runtime, libraries, and system tools.

% - **Docker Container: ** An instance of a Docker image is called a Docker container. Containers run consistently across different environments, providing a consistent and reproducible runtime.

% 2. **Kubernetes (K8s): **

% - **Container Orchestration: ** Kubernetes is an open-source container orchestration platform that automates the deployment, scaling, and management of containerized applications. It abstracts the underlying infrastructure and provides a unified API to manage clusters of containers.

% - **Key Concepts: ** Kubernetes introduces concepts like Pods (smallest deployable units), Deployments (managing replica sets and rolling updates), Services (networking abstraction for pods), and more.

% - **Scaling and Load Balancing: ** Kubernetes can scale applications horizontally by adding or removing instances (pods) based on demand. It also provides load balancing to distribute traffic across multiple instances.

% **How Docker and Kubernetes Work Together: **

% - Docker is used to create containerized applications, and Kubernetes manages the orchestration of these containers.

% - Developers package their applications into Docker containers, which can run locally on a developer's machine.

% - Kubernetes then takes these containers and orchestrates their deployment, ensuring high availability, scalability, and easy management.

% **Common Commands: **

% - **Docker Commands: **

% - `docker build`: Build a Docker image from a Dockerfile.

% - `docker run`: Create and start a Docker container.

% - `docker push`: Push a Docker image to a registry.

% - **Kubernetes Commands: **

% - `kubectl apply`: Apply configurations to a cluster.

% - `kubectl get`: Display information about resources.

% - `kubectl describe`: Show detailed information about a resource.

% - `kubectl scale`: Scale the number of replicas in a deployment.

% **Integration: **

% - Docker images are often stored in container registries like Docker Hub.

% - Kubernetes can pull these Docker images from a registry and deploy them onto the cluster.

% In summary, Docker is used to containerize applications, and Kubernetes is used to orchestrate and manage these containers in a production environment. Together, they provide a powerful and scalable solution for deploying and managing containerized applications.

% # 7. Broker Pattern dịch vụ dicovery

% https: //www.youtube.com/watch? v=UXHzxX4png0

% # 8. Dependency Injection

% # 9. Kết luận tổng kết

% Kiến trúc vi dịch vụ, với việc tách biệt hệ thống thành các thành phần nhỏ quản lý độc lập, mang lại tính linh hoạt và khả năng mở rộng.

% thiết kế hướng miền giúp xây dựng mô hình chính xác và nhất quán của lĩnh vực kinh doanh, giúp đảm bảo rằng hệ thống phản ánh đúng yêu cầu nghiệp vụ.

% <!--@============================================== -->

% <!--@============================================== -->

% <!--@============================================== -->

% <!--@============================================== -->

% <!--@============================================== -->

% <!--@============================================== -->

% <!--@============================================== -->

% <!--@============================================== -->

% <!--@============================================== -->

% <!--@============================================== -->

% <!--@============================================== -->

% <!--@saga -->

% <!--@saga -->

% <!--@saga -->

% <!--@saga -->

% <!--@saga -->

% <!--@saga -->

% <!--@saga -->

% <!--@saga -->

% <!--@saga -->

% <!--@saga -->

% <!--@saga -->

% <!--@saga -->

% <!--@saga -->

% <!--@saga -->

% <!--@saga -->

% <!--@saga -->

% <!--@saga -->

% <!--@saga -->

% <!--@saga -->

% <!--@saga -->

% <!---->

% <!--!-->

% <!--@CQRS (Command Query Responsibility Segregation): -->

% <!--CQRS, EventSourcing, Sagas-->

% <!--@Event Sourcing: -->

% <!-- Strong Consistency : https://ddd-practitioners.com/?page_id=421 -->

% <!-- Snapshots : https://ddd-practitioners.com/snapshots -->

% <!-- Saga : https://ddd-practitioners.com/home/glossary/saga -->

% <!-- Outbox Pattern -->

% <!-- Optimistic Concurrency Control : https://ddd-practitioners.com/?page_id=609 -->

% <!-- https://www.linkedin.com/pulse/api-strategy-conways-law-inverse-conway-manoeuvre-mikael-wall%C3%A9n/ -->

% Một mô hình lưu trữ dữ liệu, trong đó tất cả các thay đổi trạng thái của hệ thống được biểu diễn dưới dạng sự kiện (event).

% <!-- EventStorming : https://ddd-practitioners.com/home/glossary/eventstorming -->

% <!-- Domain Storytelling : https://ddd-practitioners.com/?page_id=1005 -->

% <!-- CQRS : https://ddd-practitioners.com/?page_id=574 -->

% CQRS chia để thoải mái, chặt chẽ

% Là một nguyên tắc trong DDD, CQRS tách biệt giữa phần xử lý câu lệnh (Command) và phần truy vấn dữ liệu (Query).

% Command đại diện cho các thao tác cập nhật dữ liệu, trong khi Query đại diện cho các thao tác truy vấn dữ liệu.

% <!-- Event-Driven Architecture : https://ddd-practitioners.com/home/glossary/event-driven-architecture -->

% <!-- Event Modeling : https://ddd-practitioners.com/?page_id=994 -->

% <!-- Event Replay : https://ddd-practitioners.com/?page_id=585 -->

% <!-- Event Sourced Aggregates : https://ddd-practitioners.com/event-sourcing -->

% <!-- Event Sourcing : https://ddd-practitioners.com/?page_id=581 -->

% <!-- Eventual Consistency : https://ddd-practitioners.com/?page_id=419 -->

% <!-- Change Data Capture: https://en.wikipedia.org/wiki/CAP_theorem -->

% <!-- ACID Transaction : https://ddd-practitioners.com/?page_id=415 -->

% ACID (Atomicity, Consistency, Isolation, Durability)

% <!-- BASE Transaction -->

% BASE là viết tắt của "Basically Available, Soft state, Eventually consistent, " và đối lập với ACID

% <!-- Command : https://ddd-practitioners.com/?page_id=596 -->

% <!-- Command Handler : https://ddd-practitioners.com/?page_id=599 -->

% <!-- Compensating Action : https://ddd-practitioners.com/compensating-action -->

% <!-- Compensating Transaction : https://ddd-practitioners.com/compensating-transaction -->

% <!-- Compensating Workflow : https://ddd-practitioners.com/compensating-workflow -->

% <!-- Domain Event : https://ddd-practitioners.com/domain-event -->

% <!--@ Dependency Inversion Principle -->

% SOLID : https://ddd-practitioners.com/home/glossary/solid

% Single Responsibility Principle : https://ddd-practitioners.com/single-responsibility-principle

% Open-Closed Principle

% Liskov Substitution Principle : https://ddd-practitioners.com/home/glossary/liskov-substitution-principle

% Interface Segregation Principle : https://ddd-practitioners.com/?page_id=817

% <!--!========================================================== -->

% <!--!========================================================== -->

% <!--!========================================================== -->

% <!--!========================================================== -->

% <!--!========================================================== -->

% <!--!========================================================== -->

% <!--!========================================================== -->

% <!-- mỗi dịch vụ xuất bản và đăng ký các sự kiện nếu cần. Cách tiếp cận này có thể mở rộng và linh hoạt hơn so với điều phối, nhưng cũng phức tạp hơn trong việc triển khai và bảo trì. Tuy nhiên, nó cũng có thể linh hoạt hơn vì mỗi dịch vụ có thể phát triển độc lập và lỗi trong một dịch vụ không nhất thiết ảnh hưởng đến toàn bộ hệ thống. -->

% <!-- PublishSubscribe : https://www.enterpriseintegrationpatterns.com/patterns/messaging/PublishSubscribeChannel.html -->

% <!--@gRPC -->

% <!--@gRPC -->

% <!--@gRPC -->

% <!--@gRPC -->

% <!--@gRPC -->

% <!--@gRPC -->

% <!--@gRPC -->

% <!--@gRPC -->

% <!--@gRPC -->

% <!--@gRPC -->

% <!--@gRPC -->

% % Để phát triển tốt

% % cần tạo một bộ kiểm thử tích hợp tự động

% CI/CD đã trình bày bên trên

% % nhằm kiểm tra tính đúng đắn

% % %! Test - Driven Development : https:// thiết kế hướng miền - practitioners.com/test - driven - development

% % %! Test - Driven Development : https:// thiết kế hướng miền - practitioners.com/test - driven - development

% % %! Test - Driven Development : https:// thiết kế hướng miền - practitioners.com/test - driven - development

% % %! Test - Driven Development : https:// thiết kế hướng miền - practitioners.com/test - driven - development

% % %! Test - Driven Development : https:// thiết kế hướng miền - practitioners.com/test - driven - development

% [[Test - Driven Development]] TDD is a lightweight programming methodology that emphasizes fast, incremental development and especially writing tests before writing code. Ideally these follow one another in cycles measured in minutes. (see full definition under [[Test - Driven Development]] topic)

% Trang chủTrang chủBảng chú giảiHướng phát triển thử nghiệm

% Hướng phát triển thử nghiệm

% Phát triển dựa trên thử nghiệm (TDD) là một phương pháp phát triển phần mềm trong đó các thử nghiệm được viết trước khi mã thực tế được phát triển. Mục đích của TDD là đảm bảo rằng mỗi đoạn mã đều được kiểm tra đầy đủ và đáp ứng các yêu cầu của doanh nghiệp. TDD liên quan đến việc viết một bài kiểm tra thất bại trước tiên, viết mã vừa đủ để vượt qua, sau đó tái cấu trúc mã để cải thiện thiết kế của nó trong khi vẫn đảm bảo rằng tất cả các bài kiểm tra đều vượt qua.

% TDD là một phương pháp quan trọng trong Thiết kế hướng miền (thiết kế hướng miền) vì nó giúp đảm bảo rằng mã được phát triển phù hợp với mô hình miền và các quy tắc miền. Bằng cách viết bài kiểm tra trước, nhà phát triển buộc phải suy nghĩ về mô hình miền và các quy tắc miền trước khi viết bất kỳ mã nào. Các thử nghiệm trở thành một cách để xác định hành vi của hệ thống và giúp tập trung vào các yêu cầu kinh doanh. Khi mã được phát triển và tái cấu trúc, các thử nghiệm sẽ đảm bảo rằng hệ thống vẫn phù hợp với mô hình miền và các yêu cầu kinh doanh.

% % %! Test - Driven Development : https:// thiết kế hướng miền - practitioners.com/test - driven - development

% % %! Test - Driven Development : https:// thiết kế hướng miền - practitioners.com/test - driven - development

% % %! Test - Driven Development : https:// thiết kế hướng miền - practitioners.com/test - driven - development

% % %! Test - Driven Development : https:// thiết kế hướng miền - practitioners.com/test - driven - development

\end{document}
%#%%%%%%%%%%%%%%%%%%%%%%%%%%%%
%!%%%%%%%%%%%%%%%%%%%%%%%%%%%%
%!%%%%%%%%%%%%%%%%%%%%%%%%%%%%
%!%%%%%%%%%%%%%%%%%%%%%%%%%%%%
%!%%%%%%%%%%%%%%%%%%%%%%%%%%%%
%!%%%%%%%%%%%%%%%%%%%%%%%%%%%%
%!%%%%%%%%%%%%%%%%%%%%%%%%%%%%
%!%%%%%%%%%%%%%%%%%%%%%%%%%%%%
%!%%%%%%%%%%%%%%%%%%%%%%%%%%%%
%!%%%%%%%%%%%%%%%%%%%%%%%%%%%%
%!%%%%%%%%%%%%%%%%%%%%%%%%%%%%
%!%%%%%%%%%%%%%%%%%%%%%%%%%%%%
%!%%%%%%%%%%%%%%%%%%%%%%%%%%%%
%!%%%%%%%%%%%%%%%%%%%%%%%%%%%%
%!%%%%%%%%%%%%%%%%%%%%%%%%%%%%
%!%%%%%%%%%%%%%%%%%%%%%%%%%%%%
%!%%%%%%%%%%%%%%%%%%%%%%%%%%%%
%!%%%%%%%%%%%%%%%%%%%%%%%%%%%%
%!%%%%%%%%%%%%%%%%%%%%%%%%%%%%
\end{document} % Kết thúc
%%%%%%%%%%%%%%%%%%%%%%%%%%%%%%
%%%%%%%%%%%%%%%%%%%%%%%%%%%%%%
%%%%%%%%%%%%%%%%%%%%%%%%%%%%%%
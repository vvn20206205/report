Các chức năng tổng quan của bài toán phụ bao gồm:

\begin{itemize}

    \item Quản lý tài khoản

          \begin{itemize}

              \item Đăng ký

                    \begin{itemize}

                        \item Để đăng ký, người nộp thuế cần chuẩn bị:

                              \begin{itemize}

                                  \item Mã số thuế

                                  \item Chữ ký số

                              \end{itemize}

                              %# Code QR

                        \item Người nộp thuế nhập mã số thuế để lấy thông tin thuế bao gồm:

                              \begin{itemize}

                                  \item Tên người nộp thuế

                                  \item Mã cơ quan thuế

                                  \item Tên cơ quan thuế

                              \end{itemize}

                              %# Code file

                              \begin{vmatrix}

                                  \begin{itemize}

                                      \item Mã số thuế có 10 ký tự đại diện cá nhân, doanh nghiệp hoặc 14 ký tự đại diện chi nhánh của doanh nghiệp với định dạng "Mã số thuế doanh nghiệp-Mã chi nhánh".

                                            \begin{example}

                                                Mã số thuế 10 ký tự: 0123456789

                                                Mã số thuế 14 ký tự: 0123456789-001

                                            \end{example}

                                            %# Code "Mã số thuế phải có độ dài 10 hoặc 14 ký tự và đúng định dạng."

                                      \item Nếu mã số thuế đã tồn tại đăng ký, hệ thống sẽ thông báo: "Mã số thuế đã đăng ký sử dụng hóa đơn điện tử."

                                            %# Code

                                  \end{itemize}
                              \end{vmatrix}

                        \item Tiếp theo, người nộp thuế nhập các thông tin:

                              \begin{itemize}

                                  \item Người liên hệ

                                  \item Điện thoại liên hệ

                                  \item Địa chỉ liên hệ

                                  \item Thư điện tử

                              \end{itemize}

                              \begin{vmatrix}

                                  \begin{itemize}

                                      \item Người liên hệ: phải chứa một chuỗi kí tự và không được để trống.

                                            %# Code

                                      \item Điện thoại liên hệ: phải chứa một chuỗi kí tự số và dấu "+" ở đầu (nếu có) và không được để trống.

                                            %# Code

                                      \item Địa chỉ liên hệ: phải chứa một chuỗi kí tự địa chỉ hợp lệ và không được để trống.

                                            %# Code

                                      \item Thư điện tử: phải chứa một chuỗi kí tự có định dạng email và không được để trống.

                                            %# Code

                                  \end{itemize}
                              \end{vmatrix}

                        \item Người nộp thuế chọn hình thức hóa đơn:

                              \begin{itemize}

                                  \item Hóa đơn có mã của cơ quan thuế

                                  \item Hóa đơn không có mã của cơ quan thuế

                              \end{itemize}

                        \item Người nộp thuế chọn loại hóa đơn:

                              \begin{itemize}

                                  \item Hóa đơn điện tử giá trị gia tăng

                                  \item Hóa đơn bán hàng

                                  \item Hóa đơn bán tài sản công

                                  \item Hóa đơn bán hàng dự trữ quốc gia

                                  \item Hóa đơn khác

                              \end{itemize}

                        \item Cuối cùng, người nộp thuế dùng chữ ký số để xác nhận gửi đăng ký với ngày thực hiện là ngày đang đăng ký hóa đơn điện tử.

                              %# Code "Gửi thông tin đăng ký hóa đơn điện tử cho cơ quan thuế thành công."

                              $\Rightarrow$ \emph{Sau khi gửi đăng kí, người nộp thuế sẽ nhận được thông báo của cơ quan thuế qua thư điện tử về việc tiếp nhận và chấp nhận đăng ký.}

                    \end{itemize}

              \item Đăng nhập

                    \begin{itemize}

                        \item Trong thông báo của cơ quan thuế về việc chấp nhận đăng ký, người nộp thuế nhận được thông tin đăng nhập tài khoản của người quản trị bao gồm:

                              \begin{itemize}

                                  \item Tên đăng nhập

                                  \item Mật khẩu

                              \end{itemize}

                        \item Người nộp thuế sử dụng thông tin đăng nhập để thực hiện đăng nhập.

                              \begin{vmatrix}

                                  \begin{itemize}

                                      \item Nếu thông tin đăng nhập không chính xác, hệ thống sẽ thông báo: "Thông tin đăng nhập không chính xác."

                                            %# Code

                                  \end{itemize}
                              \end{vmatrix}

                    \end{itemize}

              \item Đăng xuất

                    \begin{itemize}

                        \item Chức năng đăng xuất tài khoản.

                    \end{itemize}

              \item Quên mật khẩu

                    \begin{itemize}

                        \item \emph{Chức năng quên mật khẩu chỉ áp dụng cho tài khoản của người quản trị.}

                        \item Người nộp thuế cung cấp tên đăng nhập.

                        \item Người nộp thuế dùng chữ ký số để gửi yêu cầu khôi phục mật khẩu.

                              $\Rightarrow$ \emph{Sau khi gửi yêu cầu, người nộp thuế sẽ nhận được mật khẩu mới qua thư điện tử.}

                    \end{itemize}

                    %# Code "Gửi yêu cầu khôi phục mật khẩu thành công."

              \item Đổi mật khẩu
                    \begin{itemize}

                        \item Để  thay đổi mật khẩu cần  cung cấp thông tin:

                              \begin{itemize}

                                  \item  Mật khẩu cũ
                                  \item  Mật khẩu mới
                                  \item Nhập lại mật khẩu mới

                              \end{itemize}
                    \end{itemize}

              \item Thay đổi thông tin

          \end{itemize}

\end{itemize}

% % Trong quá trình sử dụng hóa đơn điện tử, khi NNT muốn thay đổi đăng ký sử dụng hóa đơn, họ có thể sử dụng chức năng "Thay đổi đăng ký sử dụng hóa đơn điện tử".

% % NNT Nhập thông tin có thể thay đổi, bao gồm: Tên NNT, Người liên hệ, Điện thoại liên hệ, Địa chỉ liên hệ, Thư điện tử.

% % Cuối cùng, NNT gửi đăng ký thay đổi với thông tin "Ngày thực hiện" là ngày NNT đang đăng ký thay đổi hóa đơn điện tử.

% % Sau khi gửi thông tin thay đổi đăng ký, NNT sẽ nhận được thông báo làm việc từ cơ quan thuế qua thư điện tử về việc tiếp nhận và chấp nhận thay đổi đăng ký cho NNT.

% % \end{itemize}

% % \item \underline{QUẢN LÝ HỆ THỐNG}

% % \begin{itemize}

% % \item Quản lý vai trò (Role Management)

% % Người quản trị hệ thống (admin) là một vai trò cố định được phép sử dụng tất cả các chức năng trên Cổng điện tử.

% % Người quản trị hệ thống có thể thực hiện CRUD "Vai trò" với các thông tin bao gồm: "ID", "Tên vai trò" và "Quyền".

% % Các quyền bao gồm:

% % Thay đổi đăng ký sử dụng hóa đơn điện tử

% % Quản lý vai trò

% % Quản lý người dùng

% % Quản lí danh mục

% % Quản lí hóa đơn

% % Tra cứu hóa đơn

% % \item Quản lý người dùng (User Management)

% % Người quản trị hệ thống có thể thực hiện CRUD "Người dùng" với các thông tin bao gồm: "Tên người dùng", "Mật khẩu", "Điện thoại", "Thư điện tử" và "Vai trò".

% % \end{itemize}

% % \item \underline{QUẢN LÝ DANH MỤC}

% % \begin{itemize}

% % \item Quản lý khách hàng (Customer Management)

% % Chức năng này thực hiện CRUD "Khách hàng" có các thông tin: "Mã khách hàng", "Tên khách hàng", "Mã số thuế", "Tên NNT", "Địa chỉ", "SĐT khách hàng", Số tài khoản, Ngân hàng

% % \item Quản lý sản phẩm (Product Management)

% % Chức năng này thực hiện CRUD "Hàng hóa" có các thông tin: "Mã hàng hóa, dịch vụ", "Tên hàng hóa, dịch vụ", "Đơn vị tính", "Đơn giá", "Thuế suất".

% % \end{itemize}

% % \item \underline{QUẢN LÝ HÓA ĐƠN}

% % \begin{itemize}

% % \item Thêm hóa đơn

% % Nhập thông tin người bán: MST người bán, Tên người bán, Địa chỉ người bán, Số điện thoại người bán.

% % Nhập thông tin người mua: Mã khách hàng, Tên khách hàng, Mã số thuế, Địa chỉ khách hàng, SĐT khách hàng.

% % Nhập thông tin hàng hóa, dịch vụ: "Số thứ tự", "Mã hàng hóa, dịch vụ", "Tên hàng hóa, dịch vụ", "Đơn vị tính", "Đơn giá", "Thuế suất" và "Số lượng".

% % Hệ thống tự động tính toán:

% % - Ngày lập hóa đơn sẽ tự động là ngày hiện tại khi người lập tạo hóa đơn mới.

% % - Tổng tiền trước thuế.

% % - Tổng tiền sau thuế.

% % \item Thay thế hóa đơn

% % Chức năng này cho phép thay đổi các thông tin trong hóa đơn gốc.

% % Lưu ý:

% % - Hãy lưu trữ thông tin ID của hóa đơn thay thế trong trạng thái "Bị thay thế" của hóa đơn gốc.

% % - Hãy lưu trữ thông tin ID của hóa đơn gốc trong trạng thái "Thay thế" của hóa đơn thay thế.

% % \item Xóa hóa đơn

% % Chức năng này cho phép xóa hóa đơn và các hóa đơn thay thế liên quan.

% % \end{itemize}

% % \item \underline{TRA CỨU HÓA ĐƠN}

% % Người sử dụng có thể thực hiện tra cứu hóa đơn trên cổng thông tin điện tử theo 2 cách:

% % Cách 1: Tra cứu hóa đơn khi NNT chưa đăng nhập

% % Cách 2: Tra cứu hóa đơn khi NNT đã đăng nhập

% % \begin{itemize}

% % \item Tra cứu khi chưa đăng nhập

% % %!<! - - Tra cứu thông tin hóa đơn - - >

% % Người tra cứu nhập thông tin bao gồm: Mã số thuế người bán, Số hóa đơn, Tổng tiền thuế, Tổng tiền thanh toán, Ngày lập hóa đơn.

% % %!<! - - Kết quả: - - >

% % %!<! - - - Nếu hóa đơn điện tử không hợp lệ, hệ thống sẽ hiển thị thông báo: "Không tồn tại hóa đơn có thông tin trùng khớp với các thông tin tổ chức, cá nhân tìm kiếm”. - - >

% % %!<! - - - Nếu hóa đơn điện tử hợp lệ, hệ thống sẽ hiển thị thông báo: "Tồn tại hóa đơn có thông tin trùng khớp với các thông tin tổ chức, cá nhân tìm kiếm". - - >

% % %!<! - - - Nếu hóa đơn tìm kiếm là hóa đơn thay thế, bị thay thế hệ thống sẽ hiển thị thông tin bổ sung về hóa đơn liên quan: "Hóa đơn này là hóa đơn thay thế cho hóa đơn có ID: {{ID}}" hoặc "Hóa đơn này là hóa đơn bị thay thế của hóa đơn có ID: {{ID}}". - - >

% % %!<! - - Tra cứu thông tin "Mã số thuế" - - >

% % Người tra cứu nhập thông tin bao gồm: Mã số thuế.

% % %!<! - - Kết quả: - - >

% % %!<! - - - Nếu đã đăng kí, hệ thống sẽ hiển thị thông báo: “MST 0107001729 đã đăng ký sử dụng hóa đơn điện tử theo Nghị định 123/2020/NĐ - CP". - - >

% % %!<! - - - Nếu NNT chưa đăng kí hoặc đã đăng kí nhưng cơ quan thuế có thông báo về việc không được chấp nhận đăng kí sử dụng hóa đơn điện tử, hệ thống sẽ hiển thị thông báo: “MST 0107001728 chưa sử dụng hóa đơn điện tử theo Nghị định 123/2020/NĐ - CP". - - >

% % \item Tra cứu khi đã đăng nhập

% % Cổng điện tử hỗ trợ tra cứu 2 loại hóa đơn là hóa đơn bán ra và hóa đơn mua vào.

% % Người tra cứu nhập thông tin tra cứu bao gồm: Mã số thuế người bán, Ngày lập hóa đơn và Số hóa đơn.

% % Cổng điện tử hỗ trợ các chức năng sau: Xem thông tin hóa đơn, In hóa đơn và Xuất hóa đơn (định dạng Excel, XML, PDF).

% % \end{itemize}

% % \item \underline{GỬI PHẢN HỒI QUA THƯ ĐIỆN TỬ}

% % \begin{itemize}

% % \item Gửi thông tin của TCT đến NNT

% % %!<! - - GỬI PHẢN HỒI QUA THƯ ĐIỆN TỬ - - >

% % %!<! - - - Gửi thông tin làm việc của TCT cho yêu cầu của NNT - - >

% % %!<! - - $ NNT nhận được thư điện tử của cơ quan thuế thông báo tiếp nhận tờ khai đăng ký - - >

% % %!<! - - $ NNT nhận được thư điện tử của cơ quan thuế chấp nhận/không chấp nhận đăng ký sử dụng HĐĐT - - >

% % %!<! - - $ NNT nhận được Thông báo tài khoản sử dụng tra cứu HĐĐT trên cổng thông tin điện tử của TCT - - >

% % %!<! - - $ NNT nhận được thư điện tử của cơ quan thuế thông báo tiếp nhận tờ khai đăng ký thay đổi - - >

% % %!<! - - $ NNT nhận được thư điện tử của cơ quan thuế chấp nhận/không chấp nhận đăng ký sử dụng HĐĐT - - >

% % \end{itemize}

% % \end{itemize}

% % $\Rightarrow$ \textbf{Tóm lại, các chức năng tổng quan của bài toán phụ bao gồm:}

% % \begin{itemize}

% % \item \underline{{QUẢN LÝ TÀI KHOẢN}}

% % \begin{itemize}

% % \item Đăng ký (Sign Up)

% % \item Đăng nhập (Sign In)

% % \item Đăng xuất (Sign Out)

% % \item Quên mật khẩu (Forgot Password)

% % \item Đổi mật khẩu (Change Password)

% % \item Thay đổi thông tin (Update Information)

% % \end{itemize}

% % \item \underline{{QUẢN LÝ HỆ THỐNG}}

% % \begin{itemize}

% % \item Quản lý vai trò (Role Management)

% % \item Quản lý người dùng (User Management)

% % \end{itemize}

% % \item \underline{{QUẢN LÝ DANH MỤC}}

% % \begin{itemize}

% % \item Quản lý khách hàng (Customer Management)

% % \item Quản lý sản phẩm (Product Management)

% % \end{itemize}

% % \item \underline{{QUẢN LÝ HÓA ĐƠN}}

% % \begin{itemize}

% % \item Thêm, thay thế, xóa hóa đơn

% % \end{itemize}

% % \item \underline{{TRA CỨU HÓA ĐƠN}}

% % \begin{itemize}

% % \item Tra cứu khi chưa đăng nhập

% % \item Tra cứu khi đã đăng nhập

% % \end{itemize}

% % \item \underline{{GỬI PHẢN HỒI QUA THƯ ĐIỆN TỬ}}

% % \begin{itemize}

% % \item Gửi thông tin của TCT đến NNT

% % \end{itemize}

% % \end{itemize}
Các chức năng tổng quan của bài toán phụ bao gồm:

\begin{itemize}

\item Quản lý tài khoản (Account Management)

\begin{itemize}

\item Đăng ký (Sign Up)

Trước khi đăng ký, NNT tạo  

\emph{Người nộp thuế (Taxpayer)} nhập \emph{mã số thuế (Tax code)} để lấy \emph{thông tin thuế (Tax information)} bao gồm: \emph{Tên của người nộp thuế (Taxpayer name)}, \emph{Mã cơ quan thuế (Tax authority code)} và \emph{Tên cơ quan thuế (Tax authority name)}.

Tiếp theo, người nộp thuế nhập các thông tin: \emph{Người liên hệ (Contact person)}, \emph{Điện thoại liên hệ (Contact phone)}, \emph{Địa chỉ liên hệ (Contact address)}, \emph{Thư điện tử (Email)}.

Cuối cùng, người nộp thuế gửi đăng ký với \emph{Ngày thực hiện (Date)} là ngày đang đăng ký hóa đơn điện tử.

\item Đăng nhập (Sign In)

Sau khi đăng ký, người nộp thuế sẽ nhận được thông báo của TCT qua thư điện tử về việc tiếp nhận và chấp nhận đăng ký bao gồm thông tin: \emph{Tên đăng nhập (Username)} và \emph{Mật khẩu (Password)} để thực hiện việc đăng nhập.

\emph{xxxxxxxxxxxxx}

\end{itemize}

\item xxxxxxxxxx

\begin{itemize}

\item xxxxxxxxxx

\item xxxxxxxxxx

\end{itemize}

\end{itemize}

% % \textbf{Em đã bỏ qua hình thức hóa đơn}

% % Hóa đơn có mã của cơ quan thuế

% % Hóa đơn không có mã của cơ quan thuế

% % \textbf{Bỏ qua các loại hóa đơn khác nhau}

% % Hóa đơn điện tử giá trị gia tăng

% % Hóa đơn bán hàng

% % Hóa đơn bán tài sản công

% % Hóa đơn bán hàng dự trữ quốc gia

% % Hóa đơn khác

% % Phiếu xuất kho kiêm vận chuyển nội bộ

% % Phiếu xuất kho gửi bán hàng đại lý

% % \end{document}

% % \textbf{Bỏ qua phần ký số}

% % USB Token hay còn gọi là chữ ký số Token là một thiết bị mà mọi doanh nghiệp, tổ chức hiện nay đều cần phải có để thực hiện khai báo và nộp thuế điện tử, cũng như để giao dịch với khách hàng.

% % \textbf{Bỏ qua phần ký hiệu hóa đơn}

% % Vì mục đích của ký hiệu hóa đơn là nhóm 6 ký tự thể hiện thông tin về loại hóa đơn điện tử có mã hoặc không mã, năm lập hóa đơn, loại hóa đơn.

% % \textbf{Bỏ qua chức năng lập hóa đơn điều chỉnh}

% % E bỏ qua chức năng lập hóa đơn điều chỉnh và chỉ có chức năng lập hóa đơn thay thế.

% % \textbf{Bỏ qua chức năng phê duyệt hóa đơn}

% % \textbf{Bỏ qua định dạng file XML, PDF, HTML, EXCEL}

% %%%%%%%%%%%%%%%%%%%%%%%%%%%%%%

% %%%%%%%%%%%%%%%%%%%%%%%%%%%%%%

% %%%%%%%%%%%%%%%%%%%%%%%%%%%%%%

% %%%%%%%%%%%%%%%%%%%%%%%%%%%%%%

% %%%%%%%%%%%%%%%%%%%%%%%%%%%%%%

% % Cả 2 cái này phj thuộc vào cái chi tiết sau?

% % Cả 2 cái này phj thuộc vào cái chi tiết sau?

% % Cả 2 cái này phj thuộc vào cái chi tiết sau?

% % Cả 2 cái này phj thuộc vào cái chi tiết sau?

% % Cả 2 cái này phj thuộc vào cái chi tiết sau?

% % Cả 2 cái này phj thuộc vào cái chi tiết sau?

% % Cả 2 cái này phj thuộc vào cái chi tiết sau?

% % Cả 2 cái này phj thuộc vào cái chi tiết sau?

% % Cả 2 cái này phj thuộc vào cái chi tiết sau?

% % Cả 2 cái này phj thuộc vào cái chi tiết sau?

% % Cả 2 cái này phj thuộc vào cái chi tiết sau?

% % Cả 2 cái này phj thuộc vào cái chi tiết sau?

% % Cả 2 cái này phj thuộc vào cái chi tiết sau?

% % Cả 2 cái này phj thuộc vào cái chi tiết sau?

% % Cả 2 cái này phj thuộc vào cái chi tiết sau?

% % Cả 2 cái này phj thuộc vào cái chi tiết sau?

% % Cả 2 cái này phj thuộc vào cái chi tiết sau?

% % Cả 2 cái này phj thuộc vào cái chi tiết sau?

% % Cả 2 cái này phj thuộc vào cái chi tiết sau?

% % Cả 2 cái này phj thuộc vào cái chi tiết sau?

% % Cả 2 cái này phj thuộc vào cái chi tiết sau?

% % Cả 2 cái này phj thuộc vào cái chi tiết sau?

% % Cả 2 cái này phj thuộc vào cái chi tiết sau?

% % Cả 2 cái này phj thuộc vào cái chi tiết sau?

% % Cả 2 cái này phj thuộc vào cái chi tiết sau?

% % Cả 2 cái này phj thuộc vào cái chi tiết sau?

% % Cả 2 cái này phj thuộc vào cái chi tiết sau?

% % Cả 2 cái này phj thuộc vào cái chi tiết sau?

% % Cả 2 cái này phj thuộc vào cái chi tiết sau?

% % Cả 2 cái này phj thuộc vào cái chi tiết sau?

% % Cả 2 cái này phj thuộc vào cái chi tiết sau?

% % Cả 2 cái này phj thuộc vào cái chi tiết sau?

% % Cả 2 cái này phj thuộc vào cái chi tiết sau?

% % Cả 2 cái này phj thuộc vào cái chi tiết sau?

% % Cả 2 cái này phj thuộc vào cái chi tiết sau?

% % Cả 2 cái này phj thuộc vào cái chi tiết sau?

% % Cả 2 cái này phj thuộc vào cái chi tiết sau?

% % Cả 2 cái này phj thuộc vào cái chi tiết sau?

% % Cả 2 cái này phj thuộc vào cái chi tiết sau?

% % Cả 2 cái này phj thuộc vào cái chi tiết sau?

% % Cả 2 cái này phj thuộc vào cái chi tiết sau?

% % Cả 2 cái này phj thuộc vào cái chi tiết sau?

% % Cả 2 cái này phj thuộc vào cái chi tiết sau?

% % Cả 2 cái này phj thuộc vào cái chi tiết sau?

% % Cả 2 cái này phj thuộc vào cái chi tiết sau?

% % Cả 2 cái này phj thuộc vào cái chi tiết sau?

% % Cả 2 cái này phj thuộc vào cái chi tiết sau?

% %%%%%%%%%%%%%%%%%%%%%%%%%%%%%%

% %%%%%%%%%%%%%%%%%%%%%%%%%%%%%%

% %%%%%%%%%%%%%%%%%%%%%%%%%%%%%%

% %%%%%%%%%%%%%%%%%%%%%%%%%%%%%%

% %%%%%%%%%%%%%%%%%%%%%%%%%%%%%%

% $\Rightarrow$ \textbf{Tóm lại, các chức năng tổng quan của bài toán phụ bao gồm:}

% \begin{itemize}

% \item \underline{{QUẢN LÝ TÀI KHOẢN}}

% \begin{itemize}

% \item Đăng ký (Sign Up)

% \item Đăng nhập (Sign In)

% \item Đăng xuất (Sign Out)

% \item Quên mật khẩu (Forgot Password)

% \item Đổi mật khẩu (Change Password)

% \item Thay đổi thông tin (Update Information)

% \end{itemize}

% \item \underline{{QUẢN LÝ HỆ THỐNG}}

% \begin{itemize}

% \item Quản lý vai trò (Role Management)

% \item Quản lý người dùng (User Management)

% \end{itemize}

% \item \underline{{QUẢN LÝ DANH MỤC}}

% \begin{itemize}

% \item Quản lý khách hàng (Customer Management)

% \item Quản lý sản phẩm (Product Management)

% \end{itemize}

% \item \underline{{QUẢN LÝ HÓA ĐƠN}}

% \begin{itemize}

% \item Thêm, thay thế, xóa hóa đơn

% \end{itemize}

% \item \underline{{TRA CỨU HÓA ĐƠN}}

% \begin{itemize}

% \item Tra cứu khi chưa đăng nhập

% \item Tra cứu khi đã đăng nhập

% \end{itemize}

% \item \underline{{GỬI PHẢN HỒI QUA THƯ ĐIỆN TỬ}}

% \begin{itemize}

% \item Gửi thông tin của TCT đến NNT

% \end{itemize}

% \end{itemize}

% \subsubsection{Chi tiết các chức năng của bài toán phụ}

% % Các yêu cầu chức năng

% \textbf{Quản lý tài khoản (Account Management)}

% \emph{Đăng ký (Sign Up)}

% Người nộp thuế nhập mã số thuế để lấy thông tin đăng ký thuế của người nộp thuế bao gồm: "Tên của người nộp thuế", "Mã cơ quan thuế quản lý" và "Tên cơ quan thuế quản lý".

% Nếu 10

% Nếu 14

% Nếu đã

% Nếu chưa

% \end{document}

% \end{document}

% \begin{itemize}

% % có 10 ký tự cho cá nhân, doanh nghiệp hoặc 14 ký tự cho chi nhánh của doanh nghiệp với định dạng "Mã số thuế doanh nghiệp - Mã chi nhánh".

% % Hệ thống hiển thị Thông tin

% % Tiếp theo, NNT nhập các thông tin hợp lệ: "Người liên hệ", "Điện thoại liên hệ", "Địa chỉ liên hệ", "Thư điện tử".

% Cuối cùng, NNT gửi đăng ký với thông tin "Ngày thực hiện" là ngày NNT đang đăng ký hóa đơn điện tử.

% Sau khi gửi thông tin đăng kí NNT sẽ nhận được thông báo làm việc của CQT qua gửi thư điện tử về việc tiếp nhận và chấp nhận đăng ký, cùng với tài khoản và mật khẩu cho NNT.

% %!<! - - // Nếu mã số thuế không đúng định dạng, hệ thống sẽ thông báo: "Mã số thuế phải có độ dài 10 hoặc 14 ký tự và đúng định dạng". - - >

% %!<! - - // Nếu mã số thuế tồn tại, hệ thống kiểm tra xem NNT đã đăng ký sử dụng hóa đơn điện tử khác chưa. Nếu đã tồn tại tờ khai đăng ký, hệ thống thông báo: "Đã tồn tại tờ khai đăng ký sử dụng hóa đơn điện tử khác của NNT đã được cơ quan thuế chấp nhận". - - >

% %!<! - - // Người liên hệ: phải chứa một chuỗi kí tự và không được để trống. - - >

% %!<! - - // Điện thoại liên hệ: phải chứa một chuỗi kí tự số và dấu " + " ở đầu chuỗi (nếu có) và không được để trống. - - >

% %!<! - - // Địa chỉ liên hệ: phải chứa một chuỗi kí tự và không được để trống. - - >

% %!<! - - // Thư điện tử: phải chứa một chuỗi kí tự có định dạng email và không được để trống. - - >

% %!<! - - // Khi NNT nhấn nút "Ký gửi", hệ thống sẽ hiển thị thông báo hỏi "Xác nhận ký gửi" với hai lựa chọn là "Đồng ý" hoặc "Hủy bỏ". - - >

% %!<! - - // Nếu NNT chọn "Đồng ý", hệ thống sẽ thông báo: "Gửi thông tin đăng ký sử dụng hóa đơn điện tử cho cơ quan thuế thành công". - - >

% \item \underline{xxxxxxxxxxxxxxxxx}

% \item Đăng nhập (Sign In)

% Sau khi CQT gửi thư điện tử chứa tài khoản và mật khẩu cho NNT, NNT thực hiện nhập đầy đủ thông tin bao gồm: Tên đăng nhập, Mật khẩu để thực hiện việc đăng nhập vào tài khoản.

% \item Đăng xuất (Sign Out)

% Chức năng để NNT đăng xuất tài khoản.

% \item Quên mật khẩu (Forgot Password)

% NNT cung cấp đầy đủ thông tin bao gồm: Tên đăng nhập, Thư điện tử. Sau đó, nhấn "Quên mật khẩu" để khôi phục mật khẩu. CQT gửi mật khẩu mới về email của NNT.

% \item Đổi mật khẩu (Change Password)

% NNT cung cấp đầy đủ thông tin bao gồm: Mật khẩu cũ, Mật khẩu mới và Nhập lại mật khẩu mới để thực hiện việc thay đổi mật khẩu.

% \item Thay đổi thông tin (Update Information)

% Trong quá trình sử dụng hóa đơn điện tử, khi NNT muốn thay đổi đăng ký sử dụng hóa đơn, họ có thể sử dụng chức năng "Thay đổi đăng ký sử dụng hóa đơn điện tử".

% NNT Nhập thông tin có thể thay đổi, bao gồm: Tên NNT, Người liên hệ, Điện thoại liên hệ, Địa chỉ liên hệ, Thư điện tử.

% Cuối cùng, NNT gửi đăng ký thay đổi với thông tin "Ngày thực hiện" là ngày NNT đang đăng ký thay đổi hóa đơn điện tử.

% Sau khi gửi thông tin thay đổi đăng ký, NNT sẽ nhận được thông báo làm việc từ cơ quan thuế qua thư điện tử về việc tiếp nhận và chấp nhận thay đổi đăng ký cho NNT.

% \end{itemize}

% \item \underline{QUẢN LÝ HỆ THỐNG}

% \begin{itemize}

% \item Quản lý vai trò (Role Management)

% Người quản trị hệ thống (admin) là một vai trò cố định được phép sử dụng tất cả các chức năng trên Cổng điện tử.

% Người quản trị hệ thống có thể thực hiện CRUD "Vai trò" với các thông tin bao gồm: "ID", "Tên vai trò" và "Quyền".

% Các quyền bao gồm:

% Thay đổi đăng ký sử dụng hóa đơn điện tử

% Quản lý vai trò

% Quản lý người dùng

% Quản lí danh mục

% Quản lí hóa đơn

% Tra cứu hóa đơn

% \item Quản lý người dùng (User Management)

% Người quản trị hệ thống có thể thực hiện CRUD "Người dùng" với các thông tin bao gồm: "Tên người dùng", "Mật khẩu", "Điện thoại", "Thư điện tử" và "Vai trò".

% \end{itemize}

% \item \underline{QUẢN LÝ DANH MỤC}

% \begin{itemize}

% \item Quản lý khách hàng (Customer Management)

% Chức năng này thực hiện CRUD "Khách hàng" có các thông tin: "Mã khách hàng", "Tên khách hàng", "Mã số thuế", "Tên NNT", "Địa chỉ", "SĐT khách hàng", Số tài khoản, Ngân hàng

% \item Quản lý sản phẩm (Product Management)

% Chức năng này thực hiện CRUD "Hàng hóa" có các thông tin: "Mã hàng hóa, dịch vụ", "Tên hàng hóa, dịch vụ", "Đơn vị tính", "Đơn giá", "Thuế suất".

% \end{itemize}

% \item \underline{QUẢN LÝ HÓA ĐƠN}

% \begin{itemize}

% \item Thêm hóa đơn

% Nhập thông tin người bán: MST người bán, Tên người bán, Địa chỉ người bán, Số điện thoại người bán.

% Nhập thông tin người mua: Mã khách hàng, Tên khách hàng, Mã số thuế, Địa chỉ khách hàng, SĐT khách hàng.

% Nhập thông tin hàng hóa, dịch vụ: "Số thứ tự", "Mã hàng hóa, dịch vụ", "Tên hàng hóa, dịch vụ", "Đơn vị tính", "Đơn giá", "Thuế suất" và "Số lượng".

% Hệ thống tự động tính toán:

% - Ngày lập hóa đơn sẽ tự động là ngày hiện tại khi người lập tạo hóa đơn mới.

% - Tổng tiền trước thuế.

% - Tổng tiền sau thuế.

% \item Thay thế hóa đơn

% Chức năng này cho phép thay đổi các thông tin trong hóa đơn gốc.

% Lưu ý:

% - Hãy lưu trữ thông tin ID của hóa đơn thay thế trong trạng thái "Bị thay thế" của hóa đơn gốc.

% - Hãy lưu trữ thông tin ID của hóa đơn gốc trong trạng thái "Thay thế" của hóa đơn thay thế.

% \item Xóa hóa đơn

% Chức năng này cho phép xóa hóa đơn và các hóa đơn thay thế liên quan.

% \end{itemize}

% \item \underline{TRA CỨU HÓA ĐƠN}

% Người sử dụng có thể thực hiện tra cứu hóa đơn trên cổng thông tin điện tử theo 2 cách:

% Cách 1: Tra cứu hóa đơn khi NNT chưa đăng nhập

% Cách 2: Tra cứu hóa đơn khi NNT đã đăng nhập

% \begin{itemize}

% \item Tra cứu khi chưa đăng nhập

% %!<! - - Tra cứu thông tin hóa đơn - - >

% Người tra cứu nhập thông tin bao gồm: Mã số thuế người bán, Số hóa đơn, Tổng tiền thuế, Tổng tiền thanh toán, Ngày lập hóa đơn.

% %!<! - - Kết quả: - - >

% %!<! - - - Nếu hóa đơn điện tử không hợp lệ, hệ thống sẽ hiển thị thông báo: "Không tồn tại hóa đơn có thông tin trùng khớp với các thông tin tổ chức, cá nhân tìm kiếm”. - - >

% %!<! - - - Nếu hóa đơn điện tử hợp lệ, hệ thống sẽ hiển thị thông báo: "Tồn tại hóa đơn có thông tin trùng khớp với các thông tin tổ chức, cá nhân tìm kiếm". - - >

% %!<! - - - Nếu hóa đơn tìm kiếm là hóa đơn thay thế, bị thay thế hệ thống sẽ hiển thị thông tin bổ sung về hóa đơn liên quan: "Hóa đơn này là hóa đơn thay thế cho hóa đơn có ID: {{ID}}" hoặc "Hóa đơn này là hóa đơn bị thay thế của hóa đơn có ID: {{ID}}". - - >

% %!<! - - Tra cứu thông tin "Mã số thuế" - - >

% Người tra cứu nhập thông tin bao gồm: Mã số thuế.

% %!<! - - Kết quả: - - >

% %!<! - - - Nếu đã đăng kí, hệ thống sẽ hiển thị thông báo: “MST 0107001729 đã đăng ký sử dụng hóa đơn điện tử theo Nghị định 123/2020/NĐ - CP". - - >

% %!<! - - - Nếu NNT chưa đăng kí hoặc đã đăng kí nhưng cơ quan thuế có thông báo về việc không được chấp nhận đăng kí sử dụng hóa đơn điện tử, hệ thống sẽ hiển thị thông báo: “MST 0107001728 chưa sử dụng hóa đơn điện tử theo Nghị định 123/2020/NĐ - CP". - - >

% \item Tra cứu khi đã đăng nhập

% Cổng điện tử hỗ trợ tra cứu 2 loại hóa đơn là hóa đơn bán ra và hóa đơn mua vào.

% Người tra cứu nhập thông tin tra cứu bao gồm: Mã số thuế người bán, Ngày lập hóa đơn và Số hóa đơn.

% Cổng điện tử hỗ trợ các chức năng sau: Xem thông tin hóa đơn, In hóa đơn và Xuất hóa đơn (định dạng Excel, XML, PDF).

% \end{itemize}

% \item \underline{GỬI PHẢN HỒI QUA THƯ ĐIỆN TỬ}

% \begin{itemize}

% \item Gửi thông tin của TCT đến NNT

% %!<! - - GỬI PHẢN HỒI QUA THƯ ĐIỆN TỬ - - >

% %!<! - - - Gửi thông tin làm việc của TCT cho yêu cầu của NNT - - >

% %!<! - - $ NNT nhận được thư điện tử của CQT thông báo tiếp nhận tờ khai đăng ký - - >

% %!<! - - $ NNT nhận được thư điện tử của CQT chấp nhận/không chấp nhận đăng ký sử dụng HĐĐT - - >

% %!<! - - $ NNT nhận được Thông báo tài khoản sử dụng tra cứu HĐĐT trên cổng thông tin điện tử của TCT - - >

% %!<! - - $ NNT nhận được thư điện tử của CQT thông báo tiếp nhận tờ khai đăng ký thay đổi - - >

% %!<! - - $ NNT nhận được thư điện tử của CQT chấp nhận/không chấp nhận đăng ký sử dụng HĐĐT - - >

% \end{itemize}

% \end{itemize}
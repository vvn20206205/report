Khi một bối cảnh bị giới hạn đã được xác định, chúng ta cần đảm bảo rằng bối cảnh bị giới hạn luôn ở trạng thái mới và hoạt động tốt, tránh xảy ra xung đột. Từ đó, đáp ứng nhu cầu doanh nghiệp phát triển thay đổi liên tục và nhanh chóng không phụ thuộc vào các múi thời gian khác nhau trên thế giới.

\emph{Tích hợp liên tục (Continuous Integration)} là công nghệ tích hợp mã nguồn liên tục, tự động kiểm thử giúp phát hiện và sửa lỗi sớm hơn, giảm thời gian cũng như rủi ro trong quá trình phát triển.

\begin{example} Jenkins là một công cụ tiêu biểu trong công nghệ tích hợp liên tục. \end{example}

\begin{figure}[H]

\centering

\includegraphics[scale = 0.4]{pictures/_vi_du_ve_jenkins_trong_cong_nghe_tich_hop_lien_tuc/main.png}

\caption{Ví dụ về Jenkins trong công nghệ tích hợp liên tục.}

\end{figure}


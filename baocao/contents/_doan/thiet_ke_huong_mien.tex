\section{Đôi nét về thiết kế hướng miền (Domain Driven Design)}

% Thiết kế hướng miền được Eric Evans giới thiệu trong cuốn sách \emph{"Domain Driven Design: Tackling Complexity in the Heart of Software"}. \emph{Thiết kế hướng miền (Domain Driven Design)} là một hướng tiếp cận thiết kế phần mềm tập trung vào việc hiểu rõ và mô hình hóa lĩnh vực kinh doanh của một tổ chức. Thiết kế hướng miền nhấn mạnh việc sử dụng lĩnh vực nghiệp vụ kinh doanh để thảo luận và đề xuất giải pháp đáp ứng nhu cầu.

% Với nhiều phần mềm được thiết kế không tốt, phần xử lý các công việc không liên quan đến vấn đề nghiệp vụ kinh doanh như truy cập tập tin, hạ tầng mạng, cơ sở dữ liệu, \dots được lập trình trong đối tượng nghiệp vụ kinh doanh. Cách này có ưu điểm giúp tốc độ hoàn thiện phần mềm nhanh. Tuy nhiên, cách này làm dự án bị mất đi tính hướng đối tượng khó thay đổi, mở rộng hệ thống, \dots Thiết kế hướng miền cung cấp một cách để tổ chức mã nguồn và dễ dàng thích ứng với các yêu cầu thay đổi.

%%%%%%%%%%%%%%%%%%%%%%%%%%%%%%

\section{Định nghĩa về miền (Domain)}

% Hệ thống phần mềm được tạo ra để xử lý công việc trong cuộc sống hiện đại. Việc phát triển hệ thống liên kết chặt chẽ với một số khía cạnh cụ thể trong cuộc sống của chúng ta. Trong thiết kế hướng miền, \emph{miền (Domain)} đề cập đến phạm vi kiến thức và vấn đề cụ thể mà hệ thống xử lý.

% \begin{itemize}

% \item Về góc độ kinh doanh: Miền đại diện cho một lĩnh vực hoặc ngành mà doanh nghiệp hoạt động.

% \item Về góc độ hệ thống: Miền có thể coi là đại diện cho không gian vấn đề của hệ thống.

% \end{itemize}

% \begin{example} Trong đồ án này, miền được xác định là bài toán giải pháp hóa đơn điện tử. \end{example}

%%%%%%%%%%%%%%%%%%%%%%%%%%%%%%

\section{Chuyên gia miền (Domain Expert)}

% Trong thiết kế hướng miền, \emph{chuyên gia miền (Domain Expert)} là người có kiến thức và hiểu biết sâu sắc về vấn đề đang được hệ thống phần mềm giải quyết. Chuyên gia miền thể hiện chính xác vấn đề kinh doanh, đóng vai trò là nguồn thông tin cho nhóm phát triển. Trong kiến trúc vi dịch vụ, thiết kế hướng miền đảm bảo mỗi dịch vụ được thiết kế phản ánh một phần cụ thể của lĩnh vực kinh doanh. Mỗi dịch vụ được quản lí bởi một nhóm phát triển được hỗ trợ bởi các chuyên gia miền.

%%%%%%%%%%%%%%%%%%%%%%%%%%%%%%

\section{Mô hình miền (Domain Models)}

% Để tạo một phần mềm tốt, chúng ta cần phải hiểu rõ về phần mềm đó. \emph{Mô hình miền (Domain Models)} là kiến thức có tổ chức và có cấu trúc về miền phù hợp để giải quyết vấn đề kinh doanh. Mục tiêu của mô hình miền là cung cấp rõ ràng, ngắn gọn và chính xác về miền làm cơ sở để hệ thống giải quyết vấn đề kinh doanh.

% \begin{example} Trong đồ án này, mô hình miền của em bao gồm yêu cầu nghiệp vụ và các sơ đồ Use Case và sơ đồ các mẫu kỹ thuật ở phần \ref{section:cac_mau_ky_thuat}. \end{example}

%%%%%%%%%%%%%%%%%%%%%%%%%%%%%%

\section{Cốt lõi của thiết kế hướng miền}

% Thiết kế hướng miền cung cấp 2 loại mẫu:

% \begin{itemize}

% \item \emph{Các mẫu chiến lược (Strategic Patterns):} Phân chia một miền lớn và phức tạp thành các phần nhỏ hơn với ranh giới được xác định rõ ràng. Giúp phân chia một miền lớn hợp lý.

% \item \emph{Các mẫu kỹ thuật (Tactical Patterns):} Hiện thực hóa các khái niệm và qui trình trong thành phần thành các thiết kế hệ thống phần mềm. Giúp hệ thống phù hợp với kinh doanh.

% \end{itemize}

% \begin{figure}[H]

% \centering

% \includegraphics[scale = 0.5]{pictures/_tong_quan_ve_cot_loi_cua_thiet_ke_huong_mien/main.drawio.png}

% \caption{Tổng quan về cốt lõi của thiết kế hướng miền}

% \end{figure}
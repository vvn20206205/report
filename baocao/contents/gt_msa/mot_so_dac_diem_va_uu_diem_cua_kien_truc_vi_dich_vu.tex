Kiến trúc vi dịch vụ có nhiều ưu điểm, đặc biệt với các dự án có quy mô lớn và phức tạp.

\begin{itemize}

\item Kiến trúc vi dịch vụ phân chia dự án thành các dịch vụ nhỏ.

\begin{itemize}

\item Giúp việc phát triển và quản lý hệ thống dễ dàng hơn.

\item Tận dụng tài nguyên theo nhu cầu cho từng dịch vụ riêng.

\end{itemize}

\item Các dịch vụ độc lập về nghiệp vụ kinh doanh.

Các nhóm không cần hiểu sâu về mọi khía cạnh kinh doanh. Dẫn tới tốc độ phát triển và tốc độ định giá doanh nghiệp nhanh hơn.

\item Các dịch vụ độc lập về ngôn ngữ lập trình và CSDL

Ví dụ: Mỗi dịch vụ sử dụng ngôn ngữ lập trình và CSDL khác nhau như: NodeJS, Go, Python, Java, CSharp, MongoDB, SQLServer, SQLite, MySQL, PostgreSQL

\begin{figure}[H]

\centering

\includegraphics[scale = 0.3]{pictures/da_ngon_ngu_va_csdl/main.drawio.png}

\caption{Các dịch vụ độc lập về ngôn ngữ lập trình và CSDL}

\end{figure}

\begin{itemize}

\item Kiến trúc vi dịch vụ sử dụng đa ngôn ngữ và công nghệ khác nhau. Từ đó tận dụng hiệu quả thế mạnh của từng ngôn ngữ, công nghệ phù hợp nhất cho yêu cầu nghiệp vụ cụ thể.

\item Giảm chi phí và thời gian kiểm thử do ít ràng buộc.

\end{itemize}

\item Các dịch vụ độc lập về triển khai hệ thống

Mỗi dịch vụ triển khai độc lập và có thể thay đổi mà không ảnh hưởng đến các dịch vụ khác.

Giảm ràng buộc và tăng tính linh hoạt của hệ thống. Từ đó dễ dàng mở rộng hệ thống.

\item Hệ thống có khả năng chịu lỗi tăng độ tin cậy.

Do các dịch vụ độc lập, nhiều dịch vụ có thể triển khai trong cùng một khả năng kinh doanh để đảm bảo tính sẵn sàng của hệ thống.

\end{itemize}

%%%%%%%%%%%%%%%%%%%%%%%%%%%%%%%%%%%%%

% Các dịch vụ tương tác với nhau qua hạ tầng mạng.


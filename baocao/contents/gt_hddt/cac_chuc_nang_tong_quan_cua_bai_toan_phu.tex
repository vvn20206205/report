Để đơn giản hóa bài toán, các chức năng trong đồ án này đã thay đổi so với bài toán thực tế trong tài liệu hướng dẫn sử dụng cổng thông tin điện tử của TCT cho hóa đơn điện tử:

\textbf{Em đã bỏ qua hình thức hóa đơn}

Hóa đơn có mã của cơ quan thuế

Hóa đơn không có mã của cơ quan thuế

\textbf{Bỏ qua các loại hóa đơn khác nhau}

Hóa đơn điện tử giá trị gia tăng

Hóa đơn bán hàng

Hóa đơn bán tài sản công

Hóa đơn bán hàng dự trữ quốc gia

Hóa đơn khác

Phiếu xuất kho kiêm vận chuyển nội bộ

Phiếu xuất kho gửi bán hàng đại lý

\textbf{Bỏ qua phần ký số}

USB Token hay còn gọi là chữ ký số Token là một thiết bị mà mọi doanh nghiệp, tổ chức hiện nay đều cần phải có để thực hiện khai báo và nộp thuế điện tử, cũng như để giao dịch với khách hàng.

\textbf{Bỏ qua phần ký hiệu hóa đơn}

Vì mục đích của ký hiệu hóa đơn là nhóm 6 ký tự thể hiện thông tin về loại hóa đơn điện tử có mã hoặc không mã, năm lập hóa đơn, loại hóa đơn.

\textbf{Bỏ qua chức năng lập hóa đơn điều chỉnh}

E bỏ qua chức năng lập hóa đơn điều chỉnh và chỉ có chức năng lập hóa đơn thay thế.

\textbf{Bỏ qua chức năng phê duyệt hóa đơn}

\textbf{Bỏ qua định dạng file XML,PDF,HTML, EXCEL}

\textbf{Tóm lại, các chức năng tổng quan của tong - cuc - thue - demo bao gồm:}

\underline{\textsc{QUẢN LÝ TÀI KHOẢN}}

Đăng ký

% mail active

% active

Đăng nhập

Đăng xuất

Quên mật khẩu

% mail reset

% reset

Đổi mật khẩu

Thay đổi thông tin

\underline{\textsc{QUẢN LÝ HỆ THỐNG}}

Quản lý vai trò

Quản lý người dùng

\underline{\textsc{QUẢN LÝ DANH MỤC}}

Danh mục khách hàng

Danh mục hàng hóa

\underline{\textsc{QUẢN LÝ HÓA ĐƠN}}

Lập hóa đơn mới

Lập hóa đơn thay thế

Hủy hóa đơn

\underline{\textsc{TRA CỨU HÓA ĐƠN}}

Tra cứu hóa đơn khi NNT chưa đăng nhập

Tra cứu hóa đơn khi NNT đã đăng nhập

\underline{\textsc{GỬI PHẢN HỒI QUA THƯ ĐIỆN TỬ}}

Gửi thông tin của TCT đến NNT
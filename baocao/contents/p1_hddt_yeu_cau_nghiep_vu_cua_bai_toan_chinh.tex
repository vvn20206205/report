Yêu cầu nghiệp vụ của bài toán chính bao gồm:

\begin{itemize}

    \item Quản lí thông báo

          Có chức năng tìm kiếm chi tiết, tìm kiếm tất cả, thêm, sửa, xóa đối với  thông báo có các thông tin:

          \begin{itemize}

              \item Mã

              \item Tiêu đề

              \item Nội dung

              \item  Thời gian

          \end{itemize}
    \item Quản lý tài khoản
          Tương tự  như bài toán phụ gồm các chức năng sau:

          \begin{itemize}

              \item Đăng ký

              \item Đăng nhập

              \item Đăng xuất

              \item Quên mật khẩu

              \item Đổi mật khẩu

              \item Thay đổi thông tin

          \end{itemize}
    \item xxxxxxxxxx
    \item xxxxxxxxxx

\end{itemize}



% %!<! - - CẤU HÌNH EMAIL - - >

% Cấu hình bao gồm:

% Địa chỉ email

% Mật khẩu email

% Loại email gửi:

% Xác nhận tài khoản mới

% Quên mật khẩu

% Gửi thông tin hóa đơn cho khách hàng

% %!<! - - QUẢN LÝ DANH MỤC - - >

% Tương tự " TCT Demo" bao gồm:

% Danh mục khách hàng

% Danh mục hàng hóa

% %!<! - - QUẢN LÝ HỆ THỐNG - - >

% Tương tự " TCT Demo" nhưng có thêm quyền "Cấu hình Email".

% %!<! - - QUẢN LÝ HÓA ĐƠN ĐIỆN TỬ - - >

% Tương tự " TCT Demo"

% %!<! - - TRA CỨU HÓA ĐƠN - - >

% Có 3 cách tra cứu:

% Tra cứu 1 hóa đơn theo "Mã hóa đơn"

% Tra cứu tất cả hóa đơn bán ra

% Tra cứu tất cả hóa đơn mua vào

% %!<! - - BÁO CÁO VÀ PHÂN TÍCH HÓA ĐƠN - - >

% Các chức năng bao gồm:

% Số lượng hóa đơn đã sử dụng

% Tổng tiền trước thuế

% Tổng tiền sau thuế

% Tổng số tiền thuế

% Số lượng khách hàng

% Số lượng sản phẩm

% %@ %@ %@ Tự động

% Nghiệp vụ của bài toán chính

% Các chức năng của bài toán chính

% THÔNG BÁO

% CRUD thông báo có (id, tiêu đề, nội dung, thời gian)

% TÀI KHOẢN

% Sử dụng tài khoản của " TCT Demo" với các chức năng tương tự Đăng ký, Đăng nhập, Đăng xuất, Quên mật khẩu, Xem thông tin, Thay đổi thông tin, Đổi mật khẩu

% CẤU HÌNH EMAIL ĐỂ GỬI HÓA ĐƠN CHO KHÁCH HÀNG

% Địa chỉ email

% Mật khẩu email

% CHỨC NĂNG DANH MỤC

% Giống với " TCT Demo" gồm "Danh mục khách hàng" và "Danh mục hàng hóa"

% TRA CỨU HÓA ĐƠN:

% Có 3 cách tra cứu:

% Tra cứu 1 hóa đơn theo "Mã hóa đơn".

% Tra cứu tất cả hóa đơn bán ra.

% Tra cứu tất cả hóa đơn mua vào.

% BÁO CÁO VÀ PHÂN TÍCH HÓA ĐƠN

% Số lượng hóa đơn đã sử dụng

% Tổng trước thuế

% Tổng sau thuế

% Tổng số tiền thuế

% Số lượng khách hàng

% Số lượng sản phẩm

% %!<! - - - - >

% %!<! - - Phân quyền - - >

% %!<! - - Thay đổi - - >

% %!<! - - Lập hóa đơn mới - - >

% %!<! - - Tra cứu - - >

% %!<! - - mail - - >
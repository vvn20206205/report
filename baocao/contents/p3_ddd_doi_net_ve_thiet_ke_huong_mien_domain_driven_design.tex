Thiết kế hướng miền được   \emph{ Eric Evans}     giới thiệu trong cuốn sách \emph{"Domain Driven Design: Tackling Complexity in the Heart of Software"}. \emph{Thiết kế hướng miền (Domain Driven Design)} là một hướng tiếp cận thiết kế phần mềm tập trung vào việc hiểu rõ và mô hình hóa lĩnh vực kinh doanh của một tổ chức. Thiết kế hướng miền nhấn mạnh việc sử dụng lĩnh vực nghiệp vụ kinh doanh để thảo luận và đề xuất giải pháp đáp ứng nhu cầu.

Với nhiều phần mềm được thiết kế không tốt, phần xử lý các công việc không liên quan đến vấn đề nghiệp vụ kinh doanh như truy cập tập tin, hạ tầng mạng, cơ sở dữ liệu, \dots được lập trình trong đối tượng nghiệp vụ kinh doanh. Cách này có ưu điểm giúp tốc độ hoàn thiện phần mềm nhanh. Tuy nhiên, cách này làm dự án bị mất đi tính hướng đối tượng khó thay đổi, mở rộng hệ thống, \dots Thiết kế hướng miền cung cấp một cách để tổ chức mã nguồn và dễ dàng thích ứng với các yêu cầu thay đổi.


Để tạo một phần mềm tốt, chúng ta cần phải hiểu rõ về phần mềm đó. \emph{Mô hình miền (Domain Models)} là kiến thức có tổ chức và có cấu trúc về miền phù hợp để giải quyết vấn đề kinh doanh. Mục tiêu của mô hình miền là cung cấp rõ ràng, ngắn gọn và chính xác về miền làm cơ sở để hệ thống giải quyết vấn đề kinh doanh.

\begin{example} Trong đồ án này, mô hình miền của em bao gồm yêu cầu nghiệp vụ, các sơ đồ Use Case và sơ đồ các mẫu kỹ thuật ở phần \ref{section:cac_mau_ky_thuat}. \end{example}
\begin{example} Trong đồ án này, mô hình miền của em bao gồm yêu cầu nghiệp vụ, các sơ đồ Use Case và sơ đồ các mẫu kỹ thuật ở phần \ref{section:cac_mau_ky_thuat}. \end{example}





\section{Các mẫu kỹ thuật trong thiết kế hướng miền} 
\section{Yêu cầu nghiệp vụ} 
\section{Phân tích sơ đồ Use Case} 
\emph{Theo em tìm hiểu có các khái niệm và căn cứ pháp lý liên quan sau đây:}








\subsection{Hóa đơn}

\emph{Theo quy định tại khoản 1 Điều 3 Nghị định 123/2020/NĐ - CP:}








%%%%%%%%%%%%%%%%%%%%%%%%%%%%%%%%%%%%%!




Hóa đơn là chứng từ kế toán do tổ chức, cá nhân bán hàng hóa, cung cấp dịch vụ lập, ghi nhận thông tin bán hàng hóa, cung cấp dịch vụ. Hóa đơn được thể hiện theo hình thức hóa đơn điện tử hoặc hóa đơn do cơ quan thuế đặt in.



%%%%%%%%%%%%%%%%%%%%%%%%%%%%%%%%%%%%%!
\subsection{Hóa đơn điện tử}

\emph{Theo quy định tại khoản 2 Điều 3 Nghị định 123/2020/NĐ - CP:}









%%%%%%%%%%%%%%%%%%%%%%%%%%%%%%%%%%%%%!

Hóa đơn điện tử là hóa đơn có mã hoặc không có mã của cơ quan thuế được thể hiện ở dạng dữ liệu điện tử do tổ chức, cá nhân bán hàng hóa, cung cấp dịch vụ lập bằng phương tiện điện tử để ghi nhận thông tin bán hàng hóa, cung cấp dịch vụ theo quy định của pháp luật về kế toán, pháp luật về thuế, bao gồm cả trường hợp hóa đơn được khởi tạo từ máy tính tiền có kết nối chuyển dữ liệu điện tử với cơ quan thuế, trong đó:

a. Hóa đơn điện tử có mã của cơ quan thuế là hóa đơn điện tử được cơ quan thuế cấp mã trước khi tổ chức, cá nhân bán hàng hóa, cung cấp dịch vụ gửi cho người mua. Mã của cơ quan thuế trên hóa đơn điện tử bao gồm số giao dịch là một dãy số duy nhất do hệ thống của cơ quan thuế tạo ra và một chuỗi ký tự được cơ quan thuế mã hóa dựa trên thông tin của người bán lập trên hóa đơn.

b. Hóa đơn điện tử không có mã của cơ quan thuế là hóa đơn điện tử do tổ chức bán hàng hóa, cung cấp dịch vụ gửi cho người mua không có mã của cơ quan thuế.

%%%%%%%%%%%%%%%%%%%%%%%%%%%%%%%%%%%%%!



\subsection{Bắt buộc sử dụng hóa đơn điện tử từ 01/07/2022}

\emph{Theo quy định tại khoản 1 Điều 59 Nghị định 123/2020/NĐ - CP:}









%%%%%%%%%%%%%%%%%%%%%%%%%%%%%%%%%%%%%!

Nghị định này có hiệu lực thi hành kể từ ngày 01 tháng 7 năm 2022, khuyến khích cơ quan, tổ chức, cá nhân đáp ứng điều kiện về hạ tầng công nghệ thông tin áp dụng quy định về hóa đơn, chứng từ điện tử của Nghị định này trước ngày 01 tháng 7 năm 2022.

%%%%%%%%%%%%%%%%%%%%%%%%%%%%%%%%%%%%%!

% Chủ đề đồ án

$\Rightarrow$ Theo quy định, tất cả các doanh nghiệp, tổ chức và hộ kinh doanh đều bắt buộc phải chuyển từ sử dụng hóa đơn giấy sang hóa đơn điện tử bắt đầu từ tháng 07/2022. Vì vậy, nhu cầu sử dụng và xử lý hóa đơn điện tử trở nên rất lớn. Do đó ở đồ án này, em chọn chủ đề về quản lý hóa đơn điện tử.

\subsection{Qui định lưu trữ hóa đơn điện tử}

\emph{Theo quy định tại khoản 1 Điều 11 Thông tư 32/2011/TT - BTC:}









%%%%%%%%%%%%%%%%%%%%%%%%%%%%%%%%%%%%%!

Người bán, người mua hàng hoá, dịch vụ sử dụng hóa đơn điện tử để ghi sổ kế toán, lập báo cáo tài chính phải lưu trữ hóa đơn điện tử theo thời hạn quy định của Luật Kế toán. Trường hợp hóa đơn điện tử được khởi tạo từ hệ thống của tổ chức trung gian cung cấp giải pháp hóa đơn điện tử thì tổ chức trung gian này cũng phải thực hiện lưu trữ hóa đơn điện tử theo thời hạn nêu trên.

%%%%%%%%%%%%%%%%%%%%%%%%%%%%%%%%%%%%%!

\emph{Theo quy định tại khoản 5 Điều 41 Luật số 88/2015/QH13:}









%%%%%%%%%%%%%%%%%%%%%%%%%%%%%%%%%%%%%!

1. Tài liệu kế toán phải được lưu trữ theo thời hạn sau đây:

a. Ít nhất là 05 năm đối với tài liệu kế toán dùng cho quản lý, điều hành của đơn vị kế toán, gồm cả chứng từ kế toán không sử dụng trực tiếp để ghi sổ kế toán và lập báo cáo tài chính.

b. Ít nhất là 10 năm đối với chứng từ kế toán sử dụng trực tiếp để ghi sổ kế toán và lập báo cáo tài chính, sổ kế toán và báo cáo tài chính năm, trừ trường hợp pháp luật có quy định khác.

c. Lưu trữ vĩnh viễn đối với tài liệu kế toán có tính sử liệu, có ý nghĩa quan trọng về kinh tế, an ninh, quốc phòng.

%%%%%%%%%%%%%%%%%%%%%%%%%%%%%%%%%%%%%!

% Thời gian lưu trữ

$\Rightarrow$ Như vậy, hóa đơn điện tử sẽ được lưu trữ trên hệ thống hóa đơn điện tử của nhà cung cấp hoặc doanh nghiệp với thời gian lưu trữ ít nhất là 10 năm theo quy định của pháp luật.



\subsection{Một số lợi ích của hóa đơn điện tử}

\emph{Một số lợi ích của hóa đơn điện tử:}









\begin{itemize}

\item Tuân thủ các quy định về thuế và pháp luật.

\item Thể hiện tính minh bạch: bảo vệ quyền lợi của người mua và người bán.

\item Giúp tiết kiệm chi phí in ấn, lưu trữ và bảo quản.

\item Loại bỏ rủi ro cháy, hỏng hoặc mất và dễ dàng sao lưu.

\item Dễ dàng tra cứu, phát hành, quản lý, tạo báo cáo và giảm thủ tục giấy tờ.

\item Giúp theo dõi tình hình tài chính của công ty (doanh thu, chi phí, lợi nhuận).

\end{itemize}

